\chapter{Chemical Equations and Reactions}
\section{Introduction}
\begin{description}
  \item[Chemical reaction] the process by which one or more substances are
    changed into one or more different substances.
  \item[Chemical equation] a representation of a chemical reaction with symbols
    and formulae, the identies and relative amounts of reactants and products in
    a chemical reaction.  An example follows: \ce{(NH4)2Cr2O7 -> N2 + Cr2O3 +
    4H2O}
\end{description}

\section{Indications}
\begin{enumerate}
  \item Evolution of heat and/or light energy
  \item Production of a gas
  \item Formation of a percipitate
  \item Change in color
\end{enumerate}

\section{Characteristics of a Chemical Equation}
\begin{enumerate}
  \item The equation must represent known facts
  \item The equation must contain the correct formulae for the reactants and
    products
  \item The law of conservation of mass must be satisfied
\end{enumerate}

\begin{itemize}
  \item The coefficients represent relative, not absolute amounts of the
    substances in the equation.
  \item The relative masses can be determined as such (see chapter on molar
    mass)
  \item The reverse chemical equation holds the same properties and descriptors
    as specified above.
\end{itemize}

\section{Balancing}
Since the law of conservation of mass must be obeyed throughout a chemical
reaction, the two sides of the equation must be \textit{balanced}.  That is to
say, that the amount of atoms of an element on one side \textit{must} equal the
amount of atoms of that same element on the other side of the equation.

The steps to balancing a chemical equation follow below:

\begin{enumerate}
  \item Write a formula equation by substituting the correct formulae for the
    reactants and products (only if a "skeleton equation is not present)
  \item Balance the formula equation according to the Law of Conservation of
    Mass:
    \begin{enumerate}
      \item Balance different types of atoms one at a time
      \item First balance atoms of elements that are combined and only appear
        once on each side
      \item Balance polyatomic ions that appear in both sides in a single unit
        \textit{as a group}
      \item Balance the \ce{H} and \ce{O} elements last.
    \end{enumerate}
  \item Ensure that the atoms are balanced
\end{enumerate}

\section{Types of Reactions}
\begin{description}
  \item[Synthesis Reaction] occurs when two or more substances combine to form a
    new compound.
  \item[Decomposition Reaction] occurs whena  single compound undergoes a
    reaction that produces two or more simpler substances
  \item[Single-replacement reaction] occurs when one element replaces a similar
    element in a compound
  \item[Double-replacement reaction] occurs when ions of two compounds exchange
    places, usually in the presence of an aqeuous solution, to form two new
    compounds
  \item[Combustion Reaction] occurs when a substances combines with \ce{O2} to
    release a large amount of energy in the form of light and heat
\end{description}

\subsection{Synthesis Reaction}
A synthesis reaction occurs when two or more substances make one substance.  The
reaction-type is as follows: \ce{A + B -> AB}.

\subsection{Decomposition Reaction}
This is the opposite of a synthesis reaction, that is to say that one compound
breaks down into its simpler substances, usually with the input of heat or
electricity.  The reaction-type is as follows: \ce{AB -> A + B}.

\subsection{Single-replacement Reaction}
A single-replacement reaction takes one element and a compound, and swaps the
similar elements: \ce{A + BC -> C + AB}.

If a metal is present, the positive element in the metal is swapped, if
negaitve, otherwise (?).

\subsection{Double-replacement Reaction}
