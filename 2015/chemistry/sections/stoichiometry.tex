\chapter{Mass Relations and Stoichiometry}
\section{Atomic Mass}
To determine the weight of an atom, scientists developed a small unit at which
such quantities may be measured.  This unit is know as the AMU, or
\textit{atomic mass unit}.

12 AMUs are defined as the weight of one atom of the Carbon-12 isotope.  That is
to say that $1amu$ is exactly the weight of 1/12th of one atom of Carbon-12.

\subsection{Masses of Elements}
However, the numbers on the periodic table are not so round?  Why might this be?
Well, since we are accounting for isotopes, we must apply a weighted average to
all naturally occuring isotopes.  For example:

\begin{table}[h]
\begin{tabular}{lll}
\hline
          & Abundance & \textit{Mass (amu)} \\ \hline
oxygen-16 & 99.762\%  & 15.994              \\
oxygen-17 & 0.038\%   & 16.999              \\
oxygen-18 & 0.200\%   & 17.999
\end{tabular}
\end{table}

The calculation for the relative \textit{amu} of Oxygen is as follows:
$(0.99762*15.994) + (0.00038*16.999) + (0.00200*17.999) = 15.9994amu$

\section{The Mole}
Avagadro's Number (or $N_A$) is thought of a quantity, usually of atoms.  The
number of that quantity is $6.022e23$.  Therefore, since this is a quantity, the
unit is $anything/mole$.

\subsection{Molar Mass}
\subsubsection{Of Elements}
To calculate the molar mass of an element: simply look to the periodic table!
In the upper-left of the element box, you will find the relative molar mass of
the element.  That is to say, the \textit{amu} of an element is the weight in
grams of one mole of that element.  For example, $1.008mol$ of \ce{H} weighs
\ce{1g}.

\subsubsection{Of Compounds}
Simply put, the molar mass of a compound is the sum of all its parts, multiplied
individiually by each of the relative quantities.

\section{Percent Composition}
\textbf{Percent composition} of an element is the percent by mass of itself
within a compound.  For example, the molar mass of \ce{Cu2S} is 159.2.  The mass
amount of \ce{Cu}

\subsection{Empirical Formulae}
An \textit{emperical formula} consists of the sumbols for the elements combined
in a compound, with subscripts showing the smallest whole-number mole ratio of
the different atoms in the compound.

For example, the emperical formula \ce{P2O5} might lead to a chemical formula of
\ce{P4O10}, or similarily, the emp. formula \ce{C3H6O} may lead to the formula
\ce{C6H18O2}.

\subsubsection{Calculating}
To calculate an emperical formula, you must first start with the percentages of
an atom or the mass of the atom present.  Once these are determined, you may
then convert to moles, and simplify the ratio by dividing by the smallest amount
and rounding.  If you cannot round cleanly, multiple the whole ratio by a
factor.

The steps for calculating this emperical formula follow below:

\begin{enumerate}
  \item Assume $100g$s was sampled, unless given other measurements.
  \item Convert the grams to moles, unless given moles.
  \item Simplify the ratio of moles by dividing by the smallest term.
  \item Multiply by factors to arrive at cleaner ratios, if not already clean.
\end{enumerate}

\section{Stoichiometry}
Stoichiometry involves the mass relationships between reactants and products in
a chemical reaction.  IT allows you to determine the amount of additional
reactants or products given the chemical equation and just one amount.

Stoichiometry relies on the ratios of moles between different compounds in a
chemical reaction.  Consider the following reaction: \ce{Al2(SO4)3 + 3Ca(OH)2 ->
2Al(OH)3 + 3CaSO4}.  It has the following mass relations:

\begin{frame}
  \centering
  {\huge $\frac{1mol \ce{Al2(SO4)3}}{3mol \ce{Ca(OH)2}}$, $\frac{1mol
  \ce{Al2(SO4)3}}{2mol \ce{Al(OH)3}}$, $\frac{1mol \ce{Al2(SO4)3}}{3mol
  \ce{CaSO4}}$}
\end{frame}

To work with these mass relations, you can convert from mass $\rightarrow$ molar
mass $\rightarrow$ moles $\rightarrow$ mole-to-mole ratio $\rightarrow$ moles
$\rightarrow$ molar mass $\rightarrow$ mass.

\subsection{Limiting Reagent}
Sometimes when doing stoichometry, one reagent appears in a limiting amount such
that you cannot fully complete the reaction.  These substances are known as
\textit{limiting reagents}.  The substances that is not completely used up in a
reaction is referred to as an \textit{excess reagent (reactant)}.

To determine the limiting reagent, just do the reaction with both against one of
the products and figure out which makes \textit{less}.  The excess reagent will
be what is left over.  To determine the quantity of the excess reagent, use the
quantity of the limiting reagent that you originally plugged in and wire that
through.

\subsubsection{Percent Yield Calculations}
Sometimes, you are given an amount of which is supposed to occur (via
stoichiometry) and an amount that you actually produced.  To calculate this, you
use the following:

{\huge $100 * \frac{actual}{theoretical}$}
