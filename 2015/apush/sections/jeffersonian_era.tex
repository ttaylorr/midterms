\chapter{Jeffersonian Era}
\section{Revolutions}
\begin{description}
  \item[Revolution of 1800] the peaceful transfer of power from the Federalists
    to the Democratic-Republicans.
  \item[Frie's rebellion] German Americans saw the raised taxes to fund the
    Quasi-War as a threat to their liberty, and they marched to Bethlehem.
  \item[Gabriel's Rebellion] Gabriel organized a revolution of rural slaves to
    capture the Virginian governor, but the rain posponed the revolution, and
    their plans were found out.  More severe slave laws were passed as a result.
\end{description}

\section{Changing Roles for Women}
Some women begin to question what the \textit{"all men are created equal"} means
for them.  New Jersey allows women to vote beginning in 1807.  The term
republican motherhood comes into being so that the women can get educated
equally in the new public school system so they can later teach their sons.

\subsection{Seperate Spheres Ideology}
\begin{description}
  \item[Private Sphere (domain of women)] women are seen as the guardians of
    morality, protectors of purity, etc.
  \item[Public Sphere (corrupt)] commerical, capitalist, etc.
\end{description}

\subsection{Benevolent Empire}
The intersection of the public and private spheres.  These institutions were
public schools, hospitals, asylums, etc.  Women were seen as the "foot-soldiers"
of this empire.

\section{Foreign Affairs}
\subsection{British Foreign Affairs}
\begin{description}
  \item[Treaty of Paris] the treaty that ended the Revolutionary war, but had two
    issues:
    \begin{itemize}
      \item Article 4: the US agreed that British merchants could recover prewar
        debts
      \item Article 4: the US agreed that they would urge states to return the
        confiscated land and property
    \end{itemize}
  \item[Impressment] British ships stopped American ships, and captured any
    sailors that they believed were dissenters of the British (these were mostly
    false accusations.
  \item[Non-Importation Act] Jefforson's first attempt to penalize Britain for
    the treatment of these ships.  Not very effective because it did not include
    the major British imports.
  \item[Chesapeak Affair]
  \item[Embargo Act] Placed a halt to all American exports in an effor to tell
    Britain to let the US be neutral in their war.  It does not help the cause
    very much, and its main result is the destruction of the U.S. economy.
  \item[Non-Intercourse Act] Lifed the Embargo for all exports to all countries
    except France and Great Britian.  Not very effective (see above) and
    somewhat difficult to enforce.
  \item[Bacon's Bill Number 2] Incentive for France and Britain to stop the
    siezement of U.S. ships.  The U.S. would continue to trade, but would stop
    with one if they siezed ships.
\end{description}

\subsection{French Foreign Affairs}
\begin{description}
  \item[Treaty of Alliance] the U.S. and France enter into an alliance that ends
    the Quasi War.
  \item[Genet Affair] Genet came to the U.S. to promote U.S. help on the
    Napoleonic Wars.  His very presence came as a threat to U.S. neutrality.
    While he was there, he was allowed to seek asylum.
  \item[XYZ Affair] U.S. diplomats were to meet with 3 French diploamts to
    negotiate, but the diplomats (hereby: X, Y, Z) demanded bribes, much to the
    offence of the U.S. diplomats.  The lack of negotiation resulted in the
    Quasi War.
  \item[Quasi War] an undeclared naval war with the French in the Carribean.  No
    clear winner.
  \item[Convention of 1800] Ended the Quasi War, made peace with France, and
    ended the obligations of the Treaty of Alliance, allowing both countries to
    forget the XYZ affair.
  \item[The Lousiana Purchase] The U.S. bought the Lousiana territory, doubling
    the size of the country.
\end{description}

\subsection{Spanish Foreign Affairs}
\begin{description}
  \item[Pinckeney's Treaty] Peace with Spain.
\end{description}

\subsection{Alien and Sedition Acts}
\begin{description}
  \item[Naturalization Act] extension of the minimum period of residence to
    achieve citizenship from 5 to 14 years.
  \item[Alien Act] allowed the expulsion of aliens deemed dangerous during peace
    time
  \item[Alien Enemies Act] allowed the expulsion of aliens deemed dangerous
    during wartime
  \item[Sedition Act] gave the government the power to fine or imprision anyone
    who criticizes the government in speech or print.
\end{description}

\section{Expanding the Border}
\begin{description}
  \item[Rush-Bagot Treaty] 1\textsuperscript{st} disarmament treaty between the
    US and Britain, limiting naval forces on the Great Lakes.
  \item[Convention of 1818] established a good deal of the US/Canada border
  \item[Adams-Oniz Treaty] the US gains Florida, and the southern Lousiana
    purcahse border is redefined.
\end{description}

\subsection{Monroe Doctrine}
\begin{enumerate}
  \item The U.S. will not get involved in entagling alliances
  \item The U.S. will not tolerate the intervention of Europeans in the Americas
  \item The U.S. will not get involved in European affairs
\end{enumerate}

\section{Market Revolution}
Around this time, Congress begins to promote a Capitalist Economy through the
passage of several bills.

\begin{description}
  \item[LLC. Laws] promotes new business by decreasing the risk to start one
  \item[US Patent and Trademark Offices] promotes the protection of inventions
  \item[Tarrifs] prevents foreign competition
  \item[Internal Improvements] roads and such help spread information
  \item[US. Postal Service] (see above)
\end{description}

\section{Supreme Court Decisions}
\begin{description}
  \item[Ogden vs. Gibbons] the Supreme Court breaks up a monopoly on steamboat
    lines between New York and New Jersey
  \item[Dartmouth vs. Woodward] Supreme Court rules that Dartmouth College is
    all good because the contract is a sacred document
  \item[Fletcher vs. Peck] again, a ruling that implies the sacredness of
    contracts
  \item[Charles River Bridge vs. Warren River Bridge] they rule against contact
\end{description}

\section{Technological Advancement}
The two major technological advancements were the Cotton Gin (efficient
harvesting of cotton, making the crop more productive) and the use of
Interchangable parts (factories became much more efficient).

\section{Social Advancement}
Women followed traditional roles, but so the market progressed.  The Putting Out
system involved women making small parts of textiles in their homes and then
putting together the product at the end.  Kind of like factory working at home.

On the other hand, the Lowell System implied that many women worked in a factory
for pay, bringing about a sense of freedom.

\section{Changing Party Systems}
\begin{description}
  \item[Hartford Convention] the death of the Federalist party.  They decide to
    meet during the War of 1812 to discuss their disagreement.
  \item[Era of Good Feelings] the democratic-republicans held most of the
    political power.  They built the \textit{American System} which consisted of
    the following:
    \begin{itemize}
      \item 2\textsuperscript{nd} bank of the U.S.
      \item Federally-funded internal improvements
      \item Tarriffs on foreign products to protect American industry
    \end{itemize}
\end{description}

\section{2nd Great Awakening}
The major ideas of the second great awakening focused on an emphasis of human
perfectability and human free will.  If these two ideals were fulfilled, then
surely the salvation would come.

\subsection{Following Reform}
\begin{description}
  \item[Femal Moral Reform Society] reformed prostitution
  \item[Cold War Society] promoted a lack of alcohol
  \item[Martha Washington Society] promoted moderation of alcohol
  \item[Smithsonian] promoted STEM
\end{description}
