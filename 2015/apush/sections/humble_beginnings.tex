\chapter{Humble Beginnings}
\section{Founding of Jamestown}
Investors of a joint-stock company known as the Virginia company were the
funding behind the Jamestown expidition.  King James I chartered the company and
territory in the New World.  The expidition nearly failed due to a lack of
prepartion in the establishment of a new community, growth, and sustinance.

\subsection{American Indian Neighbors}
The relationship between the Jamestown settlers and the American Indians started
off positively with many gift-giving cerimonies.  However, as the settlers
required more and more food, they began to attack the local natives, resulting
in sour relationships.  This model would continue to serve as the common
American/Indian relationship.

\section{The Development of Slavery}
Slavery was used almost as a response to the settlers of the New World requiring
more labor than they themselves were able to provide.  Racial hirerarchies had
long lasted in Britain.

\subsection{Labor Shortages}
Due to the labor shortages of the New World, a system of \textit{headright} was
put in place which promised a certain amount of land in exchange for your
service to it.  It was essential in drawing many people to the New World.

\subsection{Bacon's Rebellion}
Nathanial Bacon championed the cause of the frontier farmers, thus becomining
their leader.  Bacon led a group of farmers into Jamestown, burning the homes of
elite planters and the capital building.  Bacon died of diseas during the
rebellion, and it was soon squandered.

\section{African Slave Trade}
Africans, mostly young males, were brought to costal ports from Africa, where
they were sold to European slave traders, and then transported to the New World
in horrid conditions.  This part of the journey is known as the "Middle
Passage".

\subsection{The Nature of Slavery in British North America (BNA)}
Attitudes towards slavery quickly changed in the British North American
colonies.  John Casor was among the first slaves to be declared by a civil court
as a "slave for life".  Laws were passed that gave children of slaves the status
of their mother.  White Virginians eventually came to see "blacks" and "slaves"
as equal terms.

\subsection{Resistance to Slavery}
The main fear of slavery was that it could be used as a catalyst to
violentrebellion.  This rebellion was uncommon, but attempts did occur.  Stono's
rebellion is among the most popular, leading slaves to  attack a country store,
steal weapons, and kill several slave owners.  They were all sentenced to death,
and their heads were hung aside the road.

\section{Regional Development in British North America}
The new England region was driven more by religious reasons rather than economic
gain.  These settlers were Puritans, those who were followers of the breakage
from the old Catholic Church.  Purtians took their inspiration from Calvinism,
teaching that salvation was subject to a divine plan, rather than the actions of
individuals.  Puritans lived lives of stric piety, and placed a great value on
the community, believing that it was God's wish for community members to be kind
to one another.

\subsection{The Mayflower Compact, Plymouth}
The Pilgrims, (a sub-group of the Puritans) fled England to find more hospitable
religious climates.  They sought Holland, which at the time had a strong
Calvanist presence.  They were concerned about the temptations of Holland, and
eventually set sail on the \textit{Mayflower} in 1620 following the formation of
a join-stock company.

\subsection{"A City Upon a Hill"}
King Charles I granted a charter to the Massachusetts Bay Company promoting the
establishment of a colony in the northern part of British North America, but did
not specify where.  This granted a high degree of autonomy to the colonists.
John Winthrop gave a sermon which would later come to symbolize American
Exceptionalism.

\subsection{New Hampshire}
Some Puritans moved north to New HAmpshire, but were predated by small fishing
villages.  A royal decree seperated New Hampshire from Massachusetts.

\subsection{Rhode Island}
Puritan dissenter formed Rhode Island, defined by its hallmark seperation of
Church and State.

\subsubsection{Anne Hutchinson}
Anne Hutchinson, a deep religious thinker and woman held many discussions of
religious ideals in her home, taking Puritan thought to its logical extreme.  In
1638, Winthrop tried, excommunicated, and bannished Hutchinson and her family.

\subsection{Connecticut}
Some settlers wanted to rid themselves of John Winthrop.  Hooker led a group of
these to the Connecticut River, and founded the town of Hartford.

\section{The Middle Colonies}
\subsection{Maryland}
Maryland is the first proprietary colony established by England, moving away
from granting charters to joint-stock companies.  Cecekuys Cavlert was the
proprietor of this colony, focused on the cultivation of tobacco as an export
crop, using indentured servitude and slavery to work on the fields.

\subsection{North Carolina}
Formed from seperatists of the Carolina colony.  Ecomomy closely resembled the
Chesapeake colonies.

\subsection{Pennsylvania}
William Penn recieves a large piece of land, and is famed for his high degree of
religious (and otherwise) toleration.

\subsection{New Jersey, Deleware}

\subsection{New York}
New York functions as a commerical port, but it values slavery as a centrla
position in its local economy.

\section{The Lower South}
\subsection{Carolina}
Mostly planters that had migrated from Barbados.

\subsection{Georgia}
Settled as the barrier between it and Spanish Flordia.

\section{Tensions between Great Britain and North American Colonists}
\subsection{Mercantalism}
Mercantalist theory states that countries should seek to grow an empire (ex.,
Britain) and then reap the benefits of such an empire.  This includes
redirecting some profit back to the mother country, or taxation.

Britain defined a number of Navigation Acts, seeking to define the colones as
suppliers,a nd England as a market for manufactured items.
