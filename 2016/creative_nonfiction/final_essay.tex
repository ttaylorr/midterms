\documentclass[12pt]{article}
\pagestyle{plain}

\usepackage{ifpdf}
\usepackage{mla}

\begin{document}
\begin{mla}
{Taylor}{Blau}
{Mr. German}
{Creative Nonfiction (6)}
{21 January, 2016}
{``Outsider Self-Consciousness'': A Slippery Slope}

Works of the Creative Nonfiction genre offer the reader an interesting look into
the mind of the characters portrayed within them. For instance, in both John
Krakauer's \textit{Into the Wild}, Truman Capote's \textit{In Cold Blood}, and
David Foster Wallace's ``A Supposedly Fun Thing I'll Never Do Again'', some of
the characters exhibit patterns of self-consciousness. Particularly, this
self-consciousness comes in the form that they do not belong within the context
of their surroundings. Self-consciousness of this type takes on three stages. At
first, a character begins to feel isolated, a manifestation of the notion that
they do not belong. Once a character begins to feel isolated, they begin to feel
an overwhelming sense of despair. And once a character begins to feel despair,
nothing holds them back from the inevitable chain of self and
outwardly-destructive actions they will take. In short, all characters
possessing ``outsider self-consciousness'' are afflicted by isolation and
despair, which cause them to take destructive actions.

A self-consciousness of the fact that one does not belong, first and foremost,
reveals that a character feels isolated. Chris McCandless offers a revealing
look at the feelings of isolation reflected in a self-consciousness that one
does not belong. Chris is well aware that he is unlike those around him. This
difference is primarily manifested in the far-higher moral code he holds for
himself as compared to those around him. One of the moral-codes that he holds
himself to is the necessity to use things to their fullest extent: including his
Datsun. Chris shows that he is aware of the far-higher moral code that he holds
everyone around him to when he chastises his parents for gifting him a new car,
saying, ```I can't believe they'd try and buy me a car','' (Krakauer, 21). When
Chris calls his parents out for trying to gift him a new car, he is
acknowledging the differences between his own world-view, and his parents'.
Acknowledging this difference is a direct indication of the self-consciousness
Chris possesses. He, himself, is aware of the degree to which he does not belong
within society. Moreover, true to his self-consciousness, Chris shows through
his actions that he feels incredibly isolated. His isolation comes primarily in
the form of his apparent distaste for human intimacy. Despite his lengthy
travels around the country, he develops very few lasting relationships,
especially distancing himself from those who hold a different moral-code than
his own. While Chris was ``thrilled to be [traveling], he was relieved as
well--relieved that he had again evaded the impending threat of human intimacy,
of friendship, and all of the messy baggage that comes with it'' (Krakauer, 55).
Chris's attempts to evade human intimacy directly point to the notion that he
feels recluse. When one feels they have nothing in common with those around
them, it is far easier to remain isolated than to put forth the effort to make
human connection with others. This is exactly what Chris is doing. Chris is
self-conscious of the differences in his moral-code versus those around him,
which leads him to the realization that he is incredibly isolated. His lack of
attempt to make connections with others, nay, his active refusal of it, points
directly to this fact. Yet Chris is not the only character who feels isolation.
David Foster Wallace, too, feels isolation when he becomes self-conscious of the
fact that he is far different than the cruisers around him. Wallace is
especially aware of this difference when he confides his distaste for even his
own identity, noting that "there is something incredibly \textit{bovine} about a
herd of American tourists in motion, a certain greedy placidity. I feel guilty
by perceived association'' (Wallace, 310-1). When Wallace notes that he feels
``guilty by perceived association`` he admits his own feelings that he does not
belong within the crowd. Yet unlike McCandless, who is vastly different than
everyone around him, Wallace is perceived to be incredibly similar. His fear
that he will be associated with his fellow ship-mates is an admission of his
disgust for them. Here, he is self-conscious of this fact. And this
self-consciousness leads him to take incredibly isolating action. While Wallace
spends a great deal of the time during the cruise in the confines of his room,
he, too, notes his attempts to distance himself from the perceived association
between him and his shipmates. He notes that, ``all week, I've found myself
doing everything I can to distance myself in the crew's eyes from the bovine
herd I'm part of: I eschew cameras and sunglasses and pastel Carribeanwear; I
make a big deal of carrying my own luggage and my own cafeteria tray and am
effusive in my thanks for even the slightest of service'' (Wallace, 311). This
behavior is a direct result of the self-consciousness Wallace possesses. Wallace
is not just taking subtle actions here and there to distance himself from the
rest of the passengers, he is doing so at every opportunity. And each time that
he does so, he isolates himself from his surroundings just a little bit more. It
is this pattern between self-consciousness and isolation that is the first step
in the resulting chain of events.

Once a character begins feeling the effects of isolation, he is greeted by an
overwhelming presence of despair. For instance, as a result of Perry Smith's
self-consciousness that he is far removed from not only his partner Dick, but
the rest of society, he begins to feel great despair. Perry Smith truly loses
all hope in his final moments when he exclaims to the audience, ``maybe I had
something to contribute, something --'' (Capote, 340). Here, Smith displays the
last, glimmering sense of hope that he had for himself, fading away into
oblivion. The notion that he could have had something to offer is the opposite
of despair, it is bona fide hope. In that moment, Perry displayed outwardly his
inner-thoughts: perhaps his life could have indeed been worth something. He
feels hope! But as his voice falls, the effects of his isolation show. While he
literally trails off from what he is saying, he also figuratively abandons the
idea as well. Though he is blinded during this event, his figuratively faces the
citizens of Holcomb, and becomes self-conscious of the differences between him,
and them. As he realizes this, he begins to feel first the isolation, and then
the associated despair which causes him to abandon the thought that his life had
worth. Wallace, too, feels despair when he considers himself in the context of
the other cruisers. He notes that, ``There's something about a mass-market
Luxury Cruise that's unbearably sad. [...] When all the ship's structured fun
and reassurances and gaiety ceased, I felt despair'' (Wallace, 261). Wallace's
argument here is complex, but can be broken down into its component parts. When
Wallace refers to the ``structured fun and reassurances'', he is discussing the
distractions provided by a 7NC which prevent the feelings of isolation he
possesses relative to his shipmates.  But as those distractions fade, the
reminder of his isolation is omnipotent.  This feeling, caused by his
self-conscious isolation, directly correlates to the despair felt by those who
are isolated. His despair, in this sense, is justified. Consider the situation
Wallace finds himself in: he is alone on a cruise ship, out at sea. He finds no
one to connect with, and is literally isolated from the rest of humanity (re: he
is out at sea). In this sense, he, much like Smith, feels despair in the truest
sense: ultimate isolation causing one to loose all hope.

And once a character has lost all hope, they begin to take both outwardly and
inwardly-destructive actions. This conclusion is an easy one to draw. Once one
realizes that there are no consequences to his or her actions, nothing should
matter to them anymore. Plenty of examples of this pattern can be found in the
wild: for example, high-school seniors entering their second semester, and so
on. In the works considered, Chris McCandless provides a prime example of taking
destructive action facing a world where nothing matters. Chris symbolizes this
lack of caring when he discusses his plan to leave his parents: ```[...] once
the time is right, with one abrupt swift action, I'm going to completely knock
them out of my life. I'm going to divorce them as my parents...'' (Krakauer,
64). Here, Chris discusses severe action, divorcing himself from his parents
entirely, with a remarkably nonchalant attitude. Chris, influenced by the
self-consciousness of his isolation from his parents, and the society that
surrounds him, is enabled to take destructive action with remarkable ease. In
fact, the ease with which he takes these actions is even ironic, because the
choice to divorce his parents, ultimately leads to his death. It is the fact
that Chris is isolated, and therefore feels despair, that enables him to view
even the greatest of actions as inconsequential. Perry, too, exhibits these same
patterns. When remarking on the murder of the Clutters, he compares the actual
experience of killing them, of taking the life of another human being to
"picking targets off a shooting gallery" (Capote, 291). It is no secret that
Perry is self-conscious of the disconnect between him and his partner, Dick.
Because of this self-consciousness, it can be said that Perry feels isolated,
despair, and therefore, views any action as inconsequential. So too does his
murder of the Clutters fall into this pattern of thinking. Despite his actions
having great effect (ironically, he kills himself, too), he remembers them as
having no great effect. Such is the case with both McCandless and Smith, that a
self-consciousness of disconnect leads to isolation, despair, and ultimately,
destructive action.

Through examination of these characters, it can be seen that they are all
self-conscious of just how different they are to the society that surrounds
them. Whether it is a fundamental difference in the case of Dick and Perry, or a
distaste for others in the case of Wallace, each character reveals that they
feel isolated, which in turn, leads to the same great effect: despair, and
destructive action.

% Isolated -> despair -> destruction.

% Idea: all characters are self-conscious of the idea that they do not fit in
% within the context of their surroundings.
%
% Self-consciousness that you do not belongs compels you to do things that you
% don't like and leads to destructive action.
%   - Chris McCandless is compelled to participate in a life that he does not want
%     for himself (but his parents do) and is eventually so wound up that he leaves
%     for the wilderness.
%   - Perry Smith knows that he is not a murderer and is compelled to do things
%     that he does not want to do. He is lead down a destructive path and is
%     eventually punished to the highest degree.
%
% Knowing that you don't belong can make you feel incredibly isolated.
%   - Chris McCandless cannot find anyone that is quite like him. Although he is
%     meeting all of these people in his travels, he fails to develop intimacy with
%     any one of them, leaving him incredibly lonely.
%   - David Foster Wallace is confined to his room a lot of the time, resigned
%     from many of the activities aboard the ship. Although he is surrounded by
%     people, he is alone. His table-tennis game with the DJ is a prime example of
%     this. He is totally unable to connect with him leading to his loneliness.
%
% Feeling dissociated from your "group" is despair-causing:
%   - David Foster Wallace feels despair when he sees that he is the
%     quintessential "American"
%   - Perry Smith feels dissociated from the group of people that he is talking to
%     right before he is hung, a mark of self-consciousness, and thus feels
%     despair. He speaks hopelessly and realizes that there is no point in wanting
%     any longer.

\begin{workscited}
\bibent{Wallace, David Foster. ``A Supposedly Fun Thing I'll Never Do Again.''
\textit{A Supposedly Fun Thing I'll Never Do Again}. New York: Back Bay
Books/Little, Brown and Company, 1997. 256-353. Print}

\bibent{Capote, Truman. \textit{In Cold Blood}. New York: Knopf Doubleday
Publishing Group, 1992. Print.}

\bibent{Krakauer, Jon. \textit{Into the Wild}. Westmister: Knopf Doubleday
Publishing Group, 1997. Print.}

\end{workscited}
\end{mla}
\end{document}
