\section{Compare Kierkegaard and Camus}
\subsection{Outline}
\begin{description}
  \item[Thesis] Both philosophers battle the absurd, but are nuanced in both
    their definitions of it, and the ways they suggest one deal with absurdity
    in their own lives.
  \item[Body I - Defn. of absurdity] The two define absurdity in somewhat
    different fashions.
    \begin{description}
      \item[Camus] two things that don't belong together, coming together
      \item[Kierkegaard] suspension of the ethical
    \end{description}
  \item[Body II - Instructions to battle absurdity] Yet the two have vastly
    different instruction sets on how to "battle" our absurdity-frought lives.
    \begin{description}
      \item[Camus] escape vs. revolt
      \item[Kierkegaard] leap of faith become ethical such that the abusrd is
        not abusrd. He says that you have to trust that in death, the true-self
        will be revealed because life is full of absurdity and you have to trust
        in the absolute.
        \begin{itemize}
          \item Camus eschews this idea, saying that there are no such
            absolutes, only that life is full of absurdity and revolt, not
            faith, is the only way out.
        \end{itemize}
    \end{description}
\end{description}
