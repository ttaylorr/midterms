\section{Explain absurdity}
Absurdity is designated by Camus to mean two things that do not belong together,
occurring together. One thing in and of itself cannot be absurd, things are only
absurd in their presence within each other.

\begin{description}
  \item[Escape] the notion that someone living an absurd life, seeks meaning
    that Camus believes does not exist. He/she wishes to escape their absurd
    life, and launch into another life devoid of absurdity. They cannot deal
    with the absurd meaninglessness of life, and therefore seek more from it.
  \item[Revolt] the notion that someone living an absurd life can be perfectly
    content with doing so. They "revolt" against the system by accepting what
    they have been given. Sysiphus, for example, is considered the absurd
    "hero". Although his life is absurd in its entirety, he revolts against it
    by finding happiness, the thing he is not supposed to feel, within it. He
    does not need escape, his quiet revolt is the meaning for him.
\end{description}
