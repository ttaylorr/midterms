\section{Explain Platonic-Idealism}
Platonic-idealism is a tactic for explaining the nature of things in
relationship to the forms which they are images of. Take, for example, a circle.
No one has ever actually \textit{seen} a circle, only images of one. No perfect
circle has ever actually been created. Yet, the idea of a perfect circle does
exist.

Plato says that these two are distinct. The perfect circle, the one represented
by $x^2+y^2=1$ is truly perfect: it is a circle of the truest sense. It exists
in a higher realm. Yet the circle that one can see or draw themselves, is only
in the form of a circle. It is attempting to be the perfect circle, but it is
not.

This realm, in which is contained all "perfect" things, represents the "good".
The good, which cannot be described, is an idea that, once seen, enables you to
recognize all other forms. All forms can be represented within the good.
