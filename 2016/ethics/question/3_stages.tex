\section{Explain Kierkegaard's Three Stages}
\begin{description}
  \item[The Aesthetic] doing things because they feel good. Members of this
    stage do and take courses of action for no greater reason other than they
    are pleasurable to take part in. Kierkegaard implicitly did not like members
    of this stage.
  \item[The Ethical] doing things because they are the right thing to do. This
    stage implies doing things as they are set out by religious ideals (without
    blind followership).
  \item[The Leap of Faith] doing things because a higher power dictated that you
    do them. This is Abraham going to the mountain to sacrifice his son, not
    because it is a thing that he thought he should do, it is a thing that god
    told him to do. This is the teleological suspension of the ethical: the
    notion that to become virtuous in this sense, is to suspend or supercede
    what is normally considered to be right. This overpowering is, in itself,
    absurd, but is what is supposed to be done in this stage, in this scenario,
    in order to become ethical.
\end{description}
