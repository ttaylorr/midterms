\chapter{Unit 4}
\section{First Fundamental Theorem of Calculus}
Let $f$ be a function defined $\forall x\in[a,b]$. Let $F$ be the antiderivative
of $f$ on $[a,b]$. It can be said that:
\begin{equation}
  \int_{a}^{b} f(x) dx = F(b) - F(a)
\end{equation}

\subsection{The Mean Value Theorem}
If $f$ is continuous on the closed interval $[a,b]$, then there exists a number
$x=c$ such that:
\begin{equation}
  \int_{a}^{b} f(x)dx = f(c)(b-a)
\end{equation}

\section{Second Fundamental Theorem of Calculus}
\begin{equation}
  \frac{d}{dx}\left[\int_a^x f(t) dt\right]=f(x)
\end{equation}

\begin{equation}
  \frac{d}{dx}\left[ \int_a^{u(x)} f(t) dt\right] = f(u(x))\frac{du}{dx}
\end{equation}

\section{Sigma Limit Process}
Let $c_i = a+i(\Delta{x})$, and $\Delta{x}=\frac{b-a}{n}$.

\begin{equation}
  \int_a^b f(x) dx = \lim_{\Delta x\to 0} \sum_{i=1}^{n}f(c_i)\Delta{x_i}
\end{equation}

\section{Logarithmic \& Exponential Functions}
\begin{equation}
  \int \frac{du}{u} = \ln|u| + C
\end{equation}

\begin{equation}
  \int e^u dx = du * e^u + C
\end{equation}

\section{Differential Equations}
\begin{equation}
  y=Ce^{kt}
\end{equation}

\section{Inverse Trigonometric Functions}
\begin{align}
  \int \frac{1}{a^2+u^2} du = \frac{1}{a}\arctan{\frac{u}{a}} + C \\
  \int \frac{1}{\sqrt{a^2-u^2}} du = \arcsin{\frac{u}{a}} + C &&
  \int \frac{1}{u\sqrt{u^2-a^2}} du = \frac{1}{a}\arcsec{\frac{|u|}{a}} + C
\end{align}
