\chapter{Unit 5}
\section{Area Between Two Curves}
Let $f(x)$ and $g(x)$ both be curves defined $\forall{x}\in[a,b]$. Let $A$ be
the area between these two curves.

We must always assume that $f(x)$ is the larger curve between the two. Should
this change, you must split the integral.

\begin{align}
  A &= \int_a^b [f(x)-g(x)] dx \\
    &= \int_c^d [f(y)-g(y)] dy
\end{align}

\section{Volume: The Disk Method}
\subsection{Disks}
\begin{align}
  V &= \int_a^b A(x) dx \\
    &= \pi\int_a^b [r(x)]^2 dx
\end{align}

\subsection{Washers}
\begin{align}
  V &= \int_a^b A(x) dx \\
    &= \int_a^b \pi(R^2-r^2) dx \\
    &= \pi\int_a^b (R^2-r^2) dx
\end{align}

\section{Volume: The Shell Method}
Let $r(x)$ be the distance of the slice to the axis of revolution, and $h(x)$ be
the width or height of that slice.

\begin{equation}
  V = 2\pi\int_a^b r(x)h(x) dx
\end{equation}

\section{Arc Length}
Let $L$ be the arc-length of a curve defined over an interval $[a,b]$.

\begin{equation}
  L_{\text{arc}} = \int_a^b \sqrt{1+[f'(x)]^2} dx
\end{equation}

\subsection{Surfaces}
Let $r(x)$ be the distance from the rotating arc to the axis of rotation.

\begin{equation}
  S_{\text{rev}} = 2\pi\int_a^b r(x)\sqrt{1+[f'(x)]^2} dx
\end{equation}

\section{Work}
\begin{align}
  W &= \vec{F}\cdot{\vec{D}}=FD\cos\theta \\
    &= \int_a^b F(x) dx
\end{align}

\subsection{Springs}
Some work has a special definition. For example, springs follow Hooke's Law:
\begin{equation}
  F=kd
\end{equation}

\subsection{Slabs of Liquid}
\begin{align}
  \Delta F &= \text{Volume} \cdot \text{Density} \\
           &= A\Delta{y}\rho \\
  \Delta W &= \Delta{F}\cdot x
\end{align}

\subsection{Chains and Lifting}
\begin{align}
  W &= W_c + W_{nc} \\
    &= (\text{weight}\cdot\text{distance}) + (\lambda{d}\Delta{d}) + \int_a^b
  \lambda(h-y) dy
\end{align}

\section{Moments, Centers of Mass}
\subsection{Discrete, Linear Systems}
\begin{equation}
  M_O = \frac{\sum_{i=1}^{n} m_ix_i}{\sum_{i=1}^n m_i}
\end{equation}

\subsection{Two-Dimensional Systems}
\begin{equation}
  M_y = \sum_{i=1}^{n} m_ix_i
\end{equation}

\begin{equation}
  M_x = \sum_{i=1}^{n} m_iy_i
\end{equation}

\begin{equation}
  (\bar{x}, \bar{y}) = \left( \frac{M_y}{m}, \frac{M_x}{m} \right)
\end{equation}

\subsection{Planar Lamina}
\begin{align}
  m &= Da \\
    &= \int_a^b \rho(x)[f(x)-g(x)] dx
\end{align}

\begin{equation}
  M_x = \int_a^b \rho(x)\left[ \frac{f(x)+g(x)}{2} \right]\left[ f(x)-g(x) \right] dx
\end{equation}

\begin{equation}
  M_y = \int_a^b \rho(x)\left[ f(x)-g(x) \right] dx
\end{equation}

\section{Fluid Pressure and Force}
\begin{description}
  \item[Pressure] $P=wh$
  \item[Force] $F=PA=whA$
  \item[Pascal's Principle] pressure exerted by a fluid at a depth h is
    transmitted equally in all directions.
\end{description}
