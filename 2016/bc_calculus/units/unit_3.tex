\chapter{Unit 3}
\section{Related Rates}
Related rates are a technique of differentiation that allows us to examine the
rate of change of one quantity, with respect to the rate of change of several
other quantitities.

The process for using related rates in calculus is as follows:
\begin{enumerate}
  \item Identify all of the given quantities, and the quantities to be
    determined.
  \item Write an equation involving the variables whos rates of change are
    given, or to be determined.
  \item Use the Chain Rule to differentiate both sides of the equation with
    respect to time.
  \item Substitute all known values for rates of change into the original
    equation.
\end{enumerate}

An example follows below:
\textit{Air is being pumped into a spherical balloon at a rate of
\SI{4.5}{\meter\cubed\per\minute}. Find the rate of change of the radius when the
radius is \SI{2}{\meter}.}

\begin{align*}
  \frac{dV}{dt} &= \frac{9}{2} \\
  \frac{dr}{dt} \text{when} r=2 \\
  V &= \frac{4}{3}\pi{r^3} \\
  \frac{dV}{dt} &= 4\pi{r^2}\frac{dr}{dt} \\
  \frac{dr}{dt} &= \frac{1}{4\pi(2^2)}\frac{9}{2}
\end{align*}

\section{Application of Derivatives}
\subsection{Extrema on an Interval}
Definition of extrema:
\begin{enumerate}
  \item $f(c)$ is the minimum of $f$ on $I$ when $f(c) \leq f(x) \forall x \in
    I$
  \item $f(c)$ is the maximum of $f$ on $I$ when $f(c) \geq f(x) \forall x \in
    I$
\end{enumerate}

\subsubsection{The Extreme Value Theorem}
If $f$ is continuous on a closed interval $[a,b]$ then $f$ has both a minimum
and a maximum on the closed interval.

\subsubsection{Relative Extrema}
\begin{enumerate}
  \item If there is an open interval containing $c$ on which $f(c)$ is a
    minimum then $f(c)$ is a relative minimum of $f$.
  \item If there is an open interval containing $c$ on which $f(c)$ is a
    maximum then $f(c)$ is a relative maximum of $f$.
\end{enumerate}

\subsubsection{Critical Number}
If $f$ is defined at $c$ and $f'(c)=0$ or is undefined, then $c$ is a critical
number of $f$.

If $f$ has a relative minimum or relative maximum at $x=c$, then $c$ is a
critical number of $f$.

To find critical numbers of $f$ on $[a,b]$, use the following process:
\begin{enumerate}
  \item Find the critical numbers of $f$ on $(a,b)$.
  \item Evaluate $f$ at each of those critical numbers.
  \item Evaluate $f$ at each of the endpoints of $[a,b]$.
  \item The least of these values is the minimum. The greatest is the maximum.
\end{enumerate}

\subsection{Rolle's Theorem \& Mean Value Theorem}
\subsubsection{Rolle's Theorem}
Let $f$ be a continuous function defined on a closed interval $[a,b]$. If
$f(a)=f(b)$ then there is at least one $c\in(a,b)$ such that $f'(c)=0$.

\subsubsection{Mean Value Theorem}
Let $f$ be a continuous function defined $\forall{x}\in [a,b]$, and
differentiable over $x\in(a,b)$. $c\in(a,b)\to f'(c)=\frac{f(b)-f(a)}{b-a}$.

\subsection{Increasing and Decreasing Functions}
Let $f$ be a function defined over some open interval. For all critical numbers,
the following can be said about the function:
\begin{description}
  \item[$f'(x)>0$] $f(x)$ increasing
  \item[$f'(x)<0$] $f(x)$ decreasing
  \item[$f'(x)=0$] $f(x)$ constant
\end{description}

\subsubsection{First Derivative Test}
\begin{enumerate}
  \item If $f'(x)$ changes from negative to positive at $x=c$, then there is a
    relative minumum on $f$ at $x=c$.
  \item If $f'(x)$ changes from positive to negative at $x=c$, then there is a
    relative maximum on $f$ at $x=c$.
\end{enumerate}

\subsection{Concavity}
Let $f$ be differentiable on an open interval $I$. The graph of $f$ is concave
upward on $I$ when $f'(x)$ is increasing on the interval and concave downward
when $f'(x)$ is decreasing on the interval.

\subsubsection{Test for Concavity}
Let $f$ be a function whose second derivative exists $\forall{x}\in(a,b)$.
\begin{description}
  \item[$f''(x)>0\forall{x}\in{I}$] $f$ is concave upward on $I$.
  \item[$f''(x)<0\forall{x}\in{I}$] $f$ is concave downward on $I$.
\end{description}

\subsection{Limits at Infinity}
Horizontal asymptotes of a function can be determined when the limit of that
function at a positive or negative $\infty$ is a value $L$.

Let $L$ be a horizontal asymptote of a function $f$.
\begin{equation}
  L = \lim_{x\to\pm\infty} f(x)
\end{equation}

\subsection{Curve Sketching}
To sketch a curve, implement the following process:

\begin{enumerate}
  \item determine the $x$-intercepts
  \item determine the $y$-intercepts
  \item Determine any vertical asymptotes by seeing where $f(x)$ becomes
    undefined.
  \item Determine any horizontal asymptotes by evaluating the limits at
    $\pm\infty$.
  \item Determine the critical numbers of the first derivative to sketch a
    general curve of the graph.
  \item Determine the critical numbers of the second derivative to determine
    points of inflection.
\end{enumerate}

\section{Optimization}
To optimize a problem, follow steps very similar to those described in the
section on Related Rates.
\begin{enumerate}
  \item Set up a related rates problem.
  \item Determine the feasiable domain of the problem.
  \item Evaluate the function.
  \item Determine critical numbers to find the relative maximum or minimum.
\end{enumerate}

\section{Newton's Method}
Newton's method is a convenient way to determine the zeros of a function. To
solve for a zero using Newton's method, use the following process:
\begin{enumerate}
  \item Make an initial estimate $x_1$ that is close to $c$.
  \item Determine $x_{n+1}=x_n-\frac{f(x_n)}{f'(x_n)}$.
  \item See step 2.
\end{enumerate}

\section{Differentials}
