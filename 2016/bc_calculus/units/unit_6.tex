\chapter{Unit 6}
\section{Integration by Parts}
To integrate using parts, pick a $u$ and a $dv$ from the origianl integral. The
$dv$ ideally should be something that is easy to integrate, while the $u$ should
be easy to differentiate.

Once selected, simply apply the formula to yield your answer. You may have to do
parts more than once, or have the recursive case where the original integral
comes up again, and you have to move it to one side.

\begin{equation}
  \int u dv = uv - \int v du
\end{equation}

\section{Powers of Trigonometric Expressions}
\subsection{$\int \cos^n\theta\sin^m\theta d\theta$}
\begin{description}
  \item[$n$ or $m$ odd] convert all but one of that function to the other using
    a trigonometric identity. The remaining power is the $d\theta$.
  \item[$n$, $m$ both even] use a power reducing formula.
\end{description}

\subsection{$\int \sec^n\theta\tan^m\theta d\theta$}
\begin{description}
  \item[$n$ is even] convert all but one $\sec^2\theta$ to $\tan$s using a
    trigonometric identity. The remaining $\sec^2\theta$ will be your $d\theta$.
  \item[$n$ is odd] convert all but one to $\sec\theta$, saving a
    $\sec\theta\tan\theta$.
  \item[$n=0$, $m$ is anything] convert one $\tan^2\theta$ to $\sec^2\theta-1$
\end{description}

\subsection{Base-case}
If the trigonometric power expression you have is none of the above, you have to
convert everything to $\sin$s and $\cos$s and simplify from there.

\section{Triginometric Substitution}
You can substitute three cases of integrals into a triangle and then simplify
in terms of $\theta$. Integrate with respect to $d\theta$ and then convert your
answer back in terms of $x$, $y$, or $z$ using the original triangle.

\begin{align}
  \sqrt{a^2-u^2} &\to \sin\theta = \frac{u}{a} \\
  \sqrt{a^2+u^2} &\to \tan\theta = \frac{u}{a} \\
  \sqrt{u^2-a^2} &\to \sec\theta = \frac{u}{a}
\end{align}

\section{Partial Fractions}
If you have a integral where the denominator is factorable (or can be split to
become factorable) then you can use partial fractions.

\begin{equation}
  \int \frac{p(x)}{q(x)} dx = \int {\frac{A}{x}+\frac{B}{(x+1)}+\ldots\; } dx
\end{equation}

There are a few rules when dealing with expanding the denominator to use partial
fractions.
\begin{itemize}
  \item If you have a factor in the form $(px^2+q)$, the numerator is in the
    form $(Ax+B)$.
  \item If you have a factor in the form $(px+q)$ then then numerator is in the
    form of $A$.
\end{itemize}

To solve...
\begin{enumerate}
  \item Make the denominator factor-able.
  \item Break apart the partial fraction.
  \item Create the basic equation.
  \item Isolate terms by power.
  \item Solve the system.
  \item Integrate each part.
\end{enumerate}
