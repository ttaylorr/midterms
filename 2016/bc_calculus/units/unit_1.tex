\chapter{Unit 1}
\section{Plane Curves \& Parametrics}
\subsection{Definition}
Typically, functions come in rectangular form, meaning $x$ is the independent
variable, and $y$ is dependent upon it. These functions look recognizable:
$y=\ln(x)$.

To track more things, or place a function into a seperate independent variable,
parametrics are used. When using parametrics, one function is given for every
graphed dimension. For example: $x=f(t)$, $y=f(t)$ or $x=f(\theta)$,
$y=f(\theta)$.

In the above example, $x(t)$ and $y(t)$ are the parametric equations and $t$ is
the parameter. The set of points $(x, y)$ obtained as $t$ varries over the
interval on which it is defined, is the graph of the parametric equation.

\subsection{De-parameterizing}
\begin{description}
  \item[Table] A table of values can be created where $t$ varries independently,
    and $(x(t), y(t))$ are the output values.
  \item[Algebraic Simplification] $x(t)$ can be simplified in terms of $y$ and
    substituted back into $y(t)$ to obtain a function.
  \item[Trigonometric Simplification] Trigonometric identities (such as
    $\sin^2\theta + \cos^2\theta = 1$, etc.) may be used to simplify the
    parameter functions.
\end{description}

\subsection{Parameterizing}
To obtain the parametric curve of the rectangular equation, the
following process is implemented:

\begin{enumerate}
  \item Let $x(t) = t$, and write $y(t)$ in terms of $x(t)$.
  \item Write both $x$ and $y$ in terms of $t$ and $m=\frac{dy}{dx}$.
\end{enumerate}

An example is shown below for the curve $y=1-x^2$:

\begin{align*}
  x(t) &= t\\
  y(t) &= 1-x^2\\
       &= 1-t^2\\\\
  m &= \frac{dy}{dx}=-2x\\\\
  x &= \frac{-m}{2}\\\\
  y &= 1-x^2\\
    &= 1-(\frac{-m}{2})^2\\
    &= 1-\frac{m^2}{4}
\end{align*}

\section{Parametric Equations and Calculus}
\subsection{The Derivative}
Suppose a parametric equation is given $x=f(t)$ and $y=g(t)$. The slope of that
equation is given to be:
\begin{equation}
  \frac{dy}{dx}=\frac{\frac{dy}{dt}}{\frac{dx}{dt}}
\end{equation}

The second derivative, therefore, of a parametric curve is given to be:
\begin{equation}\begin{aligned}
  \frac{d^2y}{dx^2} &=
  \frac{d}{dx}\left(\frac{\frac{dy}{dt}}{\frac{dx}{dt}}\right)\\
  &= \frac{d}{dx}\left(\frac{dy}{dx}\right)
\end{aligned}\end{equation}

\subsection{Arc Length}
Recall that the formula for arc length of a curve $h(x)$ over $[x_o, x_1]$ is:
\begin{equation}
  S = \int_{x_0}^{x_1} \sqrt{1+\left[h'(x)\right]^2} dx\\
\end{equation}

Therefore making the arc length of a parametric curve the following:
\begin{align*}
  S &= \int_{x_0}^{x_1} \sqrt{1+\left[\frac{dy}{dx}\right]^2} dx\\
    &= \int_{x_0}^{x_1} \sqrt{1+\left[\frac{dy/dt}{dx/dt}\right]^2} dx\\
    &= \int_{x_0}^{x_1} \sqrt{\frac{(dx/dt)^2+(dy/dt)^2}{(dx/dt)^2}} dx\\
    &= \int_{x_0}^{x_1}
  \sqrt{\left(\frac{dx}{dt}\right)^2+\left(\frac{dy}{dt}\right)^2} dx\\
    &= \int_{x_0}^{x_1} \sqrt{[f'(t)]^2+[g'(t)]^2} dx\\
\end{align*}

\subsection{Areas of Rotation}
The revolution about the $x$-axis is given:
\begin{equation}
  S=2\pi\int_a^b
  g(t)\sqrt{\left(\frac{dx}{dt}\right)^2+\left(\frac{dy}{dt}\right)^2} dt
\end{equation}

and the area about the $y$-axis is given:
\begin{equation}
  S=2\pi\int_a^b
  f(t)\sqrt{\left(\frac{dx}{dt}\right)^2+\left(\frac{dy}{dt}\right)^2} dt
\end{equation}

\section{Vectors in the Plane}
Vectors are directed line segments that have no origin. Vectors have both
magnitude, and direction, such that any vector is the same no matter where they
start. Vectors are denoted as follows:
$$\vec{v}=\vec{PQ}$$

\subsection{Magnitude}
The magnitude of a vector $\vec{v}$ with the components $\langle x,y\rangle$ may
be calculated as follows:
\begin{equation}
  ||\vec{v}||=\sqrt{x^2+y^2}
\end{equation}

\subsection{Operations}
Let $\vec{u}=\langle u_1,u_2\rangle$ and $\vec{v}=\langle v_1,v_2\rangle$.
\begin{enumerate}
  \item The vector sum of $\vec{u}$ and $\vec{v}$ is $\langle u_1+v_1, u_2+v_2
    \rangle$.
  \item The scalar multiple of $c$ and $\vec{v}$ is $\langle cv_1, cv_2
    \rangle$.
  \item The negative of $\vec{v}$ is $\langle -v_1, -v_2 \rangle$.
  \item The difference of $\vec{u}$ and $\vec{v}$ is $\langle u_1-v_1, u_2-v_2
    \rangle$.
\end{enumerate}

To take the unit vector of a vector $\vec{v}$ in its direction, then the
following is calculated:
\begin{equation}
  \vec{u}=\frac{\vec{v}}{||\vec{v}||}=\frac{1}{||\vec{v}||}\vec{v}
\end{equation}

There are three standard unit vectors in $\mathbb{R}^3$ space:
\begin{description}
  \item[$\hat{i}$] = $\langle 1,0,0 \rangle$
  \item[$\hat{j}$] = $\langle 0,1,0 \rangle$
  \item[$\hat{k}$] = $\langle 0,0,1 \rangle$
\end{description}

\subsection{Space Coordinates and Vectors in Space}
Points in the $\mathbb{R}^2$ plane have coordinates of the form $(x,y)$. Points
in the $\mathbb{R}^3$ space have coordinates of the form $(x,y,z)$.

The distance between two such points takes the following form:
\begin{equation}
  d=\sqrt{(x_2-x_1)^2+(y_2-y_1)^2+(z_2-z_1)^2}
\end{equation}

Vectors in $\mathbb{R}^2$ take the form $\langle x,y,z \rangle$. All of the same
properties of vectors defined in $\mathbb{R}^2$ space apply here as well.

\section{The Dot Product}
The dot product determines information about the angle between two vectors. The
dot producted is calculated between two vectors $\vec{u}=\langle u_1,u_2
\rangle$ and $\vec{v}=\langle v_1,v_2 \rangle$ as follows:

\begin{equation}
  \vec{u} \cdot \vec{v} = u_1v_1 + u_2v_2
\end{equation}

\subsection{The Angle Between Two Vectors}
The dot product gives information about the angle between two non-zero vectors.
The following theorem is presented:
\begin{equation}
  \cos\theta = \frac{\vec{u}\cdot\vec{v}}{||\vec{u}||||\vec{v}||}
\end{equation}

\subsection{Direction Cosines}
The dot product can also be used to determine the angle between a vectors
components in $\mathbb{R}^3$ space and their respective planes of interest. For
example, for the vector $\vec{v} = \langle v_1,v_2,v_3 \rangle$, we can
determine the angle between each planar axis and the components of the vector.

The process is outlined below:
\begin{enumerate}
  \item Take the dot-product of a vector and a unit vector of magnitude $1$ in
    that plane (i.e., $\hat{i}$, $\hat{j}$, or $\hat{k}$).
  \item Apply the definition of the dot-product to determine the direction
    cosine.
\end{enumerate}

\begin{align}
  \cos\alpha &= \frac{v_1}{||\vec{v}||}\\
  \cos\beta  &= \frac{v_2}{||\vec{v}||}\\
  \cos\gamma &= \frac{v_3}{||\vec{v}||}
\end{align}

\subsection{Projections}
To project $\vec{u}$ onto $\vec{v}$ is to take the length of $\vec{u}$ in the
direction of $\vec{v}$. The notation is as follows below:

\begin{equation}
  \text{proj}_{\vec{v}}\vec{u}=
  \left(\frac{\vec{u}\cdot\vec{v}}{||\vec{v}||^2}\right)\vec{v}
\end{equation}

\subsubsection{Work}
\begin{align}
  W &= ||\text{proj}_{\vec{PQ}}\vec{F}||||\vec{PQ}|| \\
    &= \vec{F} \cdot \vec{PQ}
\end{align}

\section{The Cross Product}
Let $\vec{u}=\langle u_1,u_2,u_3 \rangle$ and $\vec{v}=\langle v_1,v_2,v_3
\rangle$. The cross product of $\vec{u}$ and $\vec{v}$ is said to be:
\begin{equation}
  \vec{u}\times\vec{v} = (u_2v_3-u_3v_2)\hat{i} - (u_1v_3-u_3v_1)\hat{j} +
  (u_1v_2-u_2v_1)\hat{k}
\end{equation}

Another way to calculate the cross product is with matricies:

\begin{equation}
  \begin{aligned}
    \vec{u}\times\vec{v} &= \begin{vmatrix}
      \hat{i} & \hat{j} & \hat{k} \\
      u_1     & u_2     & u_3 \\
      v_1     & v_2     & v_3
    \end{vmatrix} \\
  &= \begin{vmatrix}
    u_2 & u_3 \\
    v_2 & v_3 \\
  \end{vmatrix}\hat{i} -
  \begin{vmatrix}
    u_1 & u_3 \\
    v_1 & v_3 \\
  \end{vmatrix}\hat{j} +
  \begin{vmatrix}
    u_1 & u_2 \\
    v_1 & v_2 \\
  \end{vmatrix}\hat{k}\\
  &= (u_2v_3-u_3v_2)\hat{i} - (u_1v_3-u_3v_1)\hat{j} + (u_1v_2-u_2v_1)\hat{k}
  \end{aligned}
\end{equation}

\subsection{Algebraic Properties}
\begin{enumerate}
  \item $\vec{u}\times\vec{v}=-(\vec{v}\times\vec{u})$
  \item
    $\vec{u}\times(\vec{v}+\vec{w})=(\vec{u}\times\vec{v})+
    (\vec{u}\times\vec{w})$
  \item
    $c(\vec{u}\times\vec{v})=(c\vec{u})\times\vec{v}=\vec{u}\times(c\vec{v})$
  \item $\vec{u}\times0=0\times\vec{u}=0$
  \item $\vec{u}\times\vec{u}=0$
  \item $\vec{u}\dot(\vec{v}\times\vec{w}=(\vec{u}\times\vec{v})\cdot\vec{w}$
\end{enumerate}

\subsection{Geometric Properties}
\begin{enumerate}
  \item $\vec{u}\times\vec{v}$ is orthagonal to both $\vec{u}$ and $\vec{v}$
  \item $||\vec{u}\times\vec{v}||=||\vec{u}||||\vec{v}||\sin\theta$
  \item $\vec{u}\times\vec{v}=0$ only if $\vec{u}=c\vec{v}$
  \item $||\vec{u}\times\vec{v}||$ is the area of a parallelogram having
    $\vec{u}$ and $\vec{v}$ as adjacent sides.
\end{enumerate}

\subsection{The Triple Scalar Product}
\begin{equation}
  \vec{u}\cdot(\vec{v}\times\vec{w})=\begin{vmatrix}
    u_1 & u_2 & u_3 \\
    v_1 & v_2 & v_3 \\
    w_1 & w_2 & w_3
  \end{vmatrix}
\end{equation}

\subsubsection{Geometric Properties}
The volume of a parallelpiped with significant edges $\vec{u}$, $\vec{v}$, and
$\vec{w}$ is defined as:
\begin{equation}
  \begin{aligned}
    V &= |\vec{u}\cdot(\vec{v}\times\vec{w})| \\
      &= ||\text{proj}_{\vec{v}\times\vec{w}}||||\vec{v}\times\vec{w}|| \\
      &= \left|\frac
        {\vec{u}\cdot(\vec{v}\times\vec{w})}
        {||\vec{v}\times\vec{w}||}\right|
        ||\vec{v}\times\vec{w}|| \\
      &= |\vec{u}\cdot(\vec{v}\times\vec{w})|
  \end{aligned}
\end{equation}

\section{Lines and Planes in Space}
\subsection{Lines}
Given a line containing the point $P(x_1,y_1,z_1)$ in space parallel to the
vector $\vec{v}=\langle a,b,c \rangle$, the vector $\vec{v}$ is the direction
vector and the numbers $a$, $b$, and $c$ are direction numbers. One can surmise
that a line in space $L$ contains all points $Q(x,y,z)$ for which
$\vec{PQ}=c\vec{v}$.

\begin{equation}
  \begin{aligned}
    \vec{PQ} &= \langle x-x_1, y-y_1, z-z_1 \rangle \\
             &= \langle at, bt, ct \rangle \\
             &= t\vec{v}
  \end{aligned}
\end{equation}

The following parametric equations are generated:
\begin{align}
  x=x_1+at \\
  y=y_1+at \\
  z=z_1+at
\end{align}

and the symmetric equation of the line can be generated by eliminating the
parameter:

\begin{equation}
  \frac{x-x_1}{a} = \frac{y-y_1}{b} = \frac{z-z_1}{c}
\end{equation}

\subsection{Planes}
Consider a plane in space containing the point $P(x_1,y_1,z_1)$ and
$Q(x_2,y_2,z_2)$. There is some vector in the plane, therefore, $\vec{v}=\langle
(x_2-x_1,y_2-y_1,x_2-z-1 \rangle$, and some other vector $\vec{n}$ such that
$\vec{v}\cdot\vec{n}=0$. Let $\vec{n}=\langle a,b,c \rangle$.

It can be said that:
\begin{equation}
  \begin{aligned}
    \vec{n}\cdot\vec{PQ} &= 0\\
    \langle a,b,c \rangle \cdot \langle x-x_1, y-y_1, z-z_1 \rangle &= 0\\
    a(x-x_1)+b(y-y_1)+c(z-z_1) &= 0\\
    ax+by+cz+d &= 0
  \end{aligned}
\end{equation}

To generate the vector $\vec{n}$, it is useful to generate two vectors in the
plane $\vec{u}$ and $\vec{v}$ and then take their cross product to obtain
$\vec{n}$ such that $\vec{u}\times\vec{v}=\vec{n}$.

The dot-product can be used to determine the angle between two planes by using
their normal vectors $\vec{n_1}$ and $\vec{n_2}$.
\begin{equation}
  \cos\theta=\frac{|\vec{n_1}\cdot\vec{n_2}|}{||\vec{n_1}||||\vec{n_2}||}
\end{equation}

\subsubsection{Distance between a Point and a Plane}
The distance $D$ between a point $Q(x_0,y_0,z_0)$ (not in the plane) and $P$ (in
the plane with the direction numbers $A$, $B$, $C$, and $D$) is:
\begin{equation}
  \begin{aligned}
    D &= ||\text{proj}_{\vec{n}}\vec{PQ}|| \\
      &= \frac{\vec{PQ}\cdot\vec{n}}{||\vec{n}||} \\
      &= \frac{|ax_0+by_0+cz_0+d|}{\sqrt{a^2+b^2+c^2}}
  \end{aligned}
\end{equation}

\subsubsection{Distance between a Point and a Line}
The distance $D$ between a point $Q$ and a line in space containing the point
$P$ is:
\begin{equation}
  D = \frac{||\vec{PQ}\times\vec{u}||}{||\vec{u}||}
\end{equation}

\section{Vectored-Valued Functions}
Vectored valued functions are functions that produced vectors ($\langle x,y,z
\rangle$ in $\mathbb{R}^3$ space, $\langle x,y \rangle$ in $\mathbb{R}^2$ space).

Vectored valued functions are denoted as follows:

\begin{equation}
  \vec{r}(t) = \langle \vec{f}(t), \vec{g}(t), \vec{h}(t) \rangle
\end{equation}

These functions are drawn as normal curves, such that each point on the curve
$P(x,y)$ represents a vector $\vec{v} = \langle x,y \rangle$. Just like normal
parametric curves, they have direction or orientation.

The limits of these functions may be taken as follows:
\begin{equation}
  \lim_{x\to{a}} \vec{r}(t) =
    \left[\lim_{x\to{a}}\vec{f}(t)\right]\hat{i} +
    \left[\lim_{x\to{a}}\vec{g}(t)\right]\hat{j} +
    \left[\lim_{x\to{a}}\vec{h}(t)\right]\hat{k}
\end{equation}

\section{Calculus of Vectored-Valued Functions}
\subsection{Differentiation}
The derivative of a vectored-valued function $\vec{r}(t)$ is given as:
\begin{equation}
  \vec{r}'(t)=\lim_{\Delta{t}\to{0}}
    \frac{\vec{r}(t+\Delta{t})-\vec{r}(t)}{\Delta{t}}
\end{equation}

Such that:
\begin{equation}
  \vec{r}'(t) = \frac{d}{dt}\vec{f}(t) +
                \frac{d}{dt}\vec{g}(t) +
                \frac{d}{dt}\vec{h}(t)
\end{equation}

\subsection{Properties}
\begin{enumerate}
  \item $$\frac{d}{dt}[\vec{r}(t)\cdot\vec{u}(t)] = \vec{r}(t)\cdot\vec{u}'(t) +
                                                    \vec{u}(t)\cdot\vec{r}'(t)$$
  \item $$\frac{d}{dt}[\vec{r}(t)\times\vec{u}(t)]=\vec{r}(t)\times\vec{u}'(t) +
                                                   \vec{u}(t)\times\vec{r}'(t)$$
\end{enumerate}

\subsection{Integration}
The integral of a vectored-valued function $\vec{r}(t)$ is given as: (note, this
applies to vector-valued functions in $\mathbb{R}^2$ and $\mathbb{R}^3$ for both
definite and indefinite integrals)
\begin{equation}
  \int \vec{r}(t) dt = \left(\int \vec{f}(t) dt\right)\hat{i} +
                       \left(\int \vec{g}(t) dt\right)\hat{j} +
                       \left(\int \vec{h}(t) dt\right)\hat{k}
\end{equation}

\section{Velocity \& Acceleration}
Let $\vec{a}(t)$, $\vec{v}(t)$ and $\vec{r}(t)$ represent the acceleration,
velocity, and position of an object in $\mathbb{R}^2$ or $\mathbb{R}^3$
respectively.
\begin{equation}
  \vec{a}(t) = \frac{d}{dt}\left[\vec{v}(t)\right]
             = \frac{d^2}{dt^2}\left[\vec{r}(t)\right]
\end{equation}
