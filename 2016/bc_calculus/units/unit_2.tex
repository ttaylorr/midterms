\chapter{Unit 2}
\section{Continuity, Intermediate Value Theroem}
\subsection{Continuity}
Let $f(x)$ be a function defined at $x=a$. $f$ is said to be continuous at $x=a$
if:
\begin{equation}
  \lim_{x\to{a^+}} f(x) = \lim_{x\to{a^-}} f(x) = f(a)
\end{equation}

\subsection{Intermediate Value Theorem}
Let $f(x)$ be defined over $x\in[a,b]$. Suppose that $f(a)$ and $f(b)$ both
exist. The intermediate value theorem states that there exists a $c\in[a,b]$
such that $f(a)<f(c)=V<f(b)$.

\section{$\delta$-$\epsilon$ proofs}
Preforming a $\delta$-$\epsilon$ proof means proving the existence of a limit
using the $\delta$-$\epsilon$ definition of a limit. That definition follows
below:

Suppose:
$$\lim_{x\to{c}} f(x) = L$$

It can be said, therefore:
\begin{equation}
  |x-c|<\delta \Rightarrow |f(x)-L|<\epsilon
\end{equation}

To prove the existence of a limit using the $\delta$-$\epsilon$ definition of
the limit, follow the process below:
\begin{enumerate}
  \item Write $|x-c|<\delta$ filling in known values of $c$.
  \item Write $|f(x|-L|<\epsilon$ filling in the values of $f(x)$ in terms of
    $x$ and $L$.
  \item Manipulate one inequality to look like the other, and set them equal.
  \item Determine the relationship between $\delta$ and $\epsilon$.
  \item Reverse to prove.
\end{enumerate}

\section{Indeterminate Forms \& L'Hopital's rule}
\subsection{Indeterminate Forms}
When evaluating a limit, it may become impossible to continue evaluation if the
inner-function of the limit is not defined.

For limits of the form:
$$\lim_{x\to{c}}\frac{p(x)}{q(x)}=L$$

If $\frac{p(x)}{q(x)}$ is one of the following forms: $\frac{0}{0}$ or
$\frac{\pm\infty}{\pm\infty}$, then L'Hopital's rule may be applied.

L'Hopital's rule states the following for limits with indeterminate forms:

\begin{equation}
  \lim_{x\to{c}} \frac{p(x)}{q(x)} = \lim_{x\to{c}} \frac{p'(x)}{q'(x)}
\end{equation}

...and so-on and so-forth until the limit exists (so long as $\frac{p(x)}{q(x)}$
is an indeterminate form).

\section{Limits and Continuity}
\section{Derivatives}
The derivative of a function is another function which is valued as the set of
instantenous rates of changes of its "parent" function. A definition follows:

\subsection{Limit Definition}
\begin{equation}
  f'(x)=\lim_{h\to{0}} \frac{f(x+h)-f(x)}{h}
\end{equation}

\subsection{Alternate Form}
The alternate form of the derivate follows:
\begin{equation}
  f'(c)=\lim_{x\to{c}} \frac{f(x)-f(c)}{x-c}
\end{equation}

\subsection{Implicit Differentation}
Differentiate an equation with respect to one variable, and other differentials
remain. You can move all differentials of a single kind over to one side to
solve. An example follows below:

\begin{align*}
  x^2+y^2 &= \sin(xy)-y^2 \\
  \frac{d}{dx}[x^2+y^2] &= \frac{d}{dt}[\sin(xy)-y^2] \\
  2x+2y\frac{dy}{dt} &= cos(xy)\left(y+x\frac{dy}{dt}\right)-2y\frac{dy}{dt}
\end{align*}

\subsection{Average Rate of Change}
The average rate of change gives the average rate of change of $f(x)$ on the
interval $[a,b]$.
\begin{equation}
  \frac{f(b)-f(a)}{b-a}
\end{equation}

\subsection{Instantenous Rate of Change}
The instantenous rate of change gives the rate of change at a particular point.
\begin{equation}
  f'(c)
\end{equation}

\subsection{Arctrig Proofs}
Suppose $y=\arctan(x)$, therefore $y'=\frac{dx}{1+x^2}$. Prove.

\begin{equation}
\begin{aligned}
  \tan(y) &= x \\
  \sec^2(y)\frac{dy}{dx} &= 1 \\
  \frac{1}{\sec^2(y)} &= \frac{dy}{dx} \\
  \frac{1}{1+\tan^2(y)} &= \\
  \frac{1}{1+x^2} &=
\end{aligned}
\end{equation}

\subsection{Differentiability}
Suppose $f(x)$ is a function such that $\{c\} \in D_{f(x)}$. While
differentiability at $x=c$ is an implication of continuity at $x=c$, the
converse is not necessairly true. Continuity implies differentiability in the
following scenarios:
\begin{itemize}
  \item $f(x)$ has normal behavior around $x=c$.
\end{itemize}

...but does not imply differentiability in the following scenarios:
\begin{itemize}
  \item $f(x)$ has a cusp at $x=c$.
  \item $f(x)$ has a vertical "jump" at $x=c$.
\end{itemize}

To prove differentiability of $f$ at a point $x=c$, the following steps must be
taken:
\begin{enumerate}
  \item $\lim_{x\to{c}} \frac{f(x)-f(c)}{x-c}$ must exist and "agree" from both
    sides.
\end{enumerate}

\subsection{Inverse Functions}
The derivative of an inverse function is defined:
\begin{equation}
  \frac{d}{dx}\left[f^{-1}(x)\right]=\frac{1}{f'(f^{-1}(x))}
\end{equation}

\section{Logarithmic and Exponential Functions}
If it is impossible to differentiate or take the limit of a function w.r.t. $x$,
it is sometimes useful to take the natural log of both sides such that
L'Hopital's rule may be applied. An example is shown below:
\begin{equation}
  \begin{aligned}
    L &= \lim_{x\to\infty} \left(1+\frac{1}{x}\right)^x \\
    \ln{L} &= \ln{\lim_{x\to\infty} \left(1+\frac{1}{x}\right)^x} \\
           &= \lim_{x\to\infty} \ln\left(\left(1+\frac{1}{x}\right)^x\right) \\
           &= \lim_{x\to\infty} x\ln\left(1+\frac{1}{x}\right) \\
           &= \lim_{x\to\infty} \frac{\ln\left(1+\frac{1}{x}\right)}{x^{-1}} \\
           &= \lim_{x\to\infty} \frac{0}{0} \\
           &= \lim_{x\to\infty} \frac{\frac{1}{1+\frac{1}{x}}x^{-2}}{-x^{-2}} \\
           &= \lim_{x\to\infty} \frac{1}{1+\frac{1}{x}} \\
           &= 1 \\
    L &= e.
  \end{aligned}
\end{equation}

\section{Derivates w.r.t Other Functions}
As we have seen so far, functions can be differentiated with respect to a
variable of that function (ex.: $f(x)$ and $\frac{d}{dx}(f(x))$).

However, it is possible to differentiate a function with respect to another
function. To differentiate a function $f(x)$ with respect to another function
$g(x)$, the following equation is used:
\begin{equation}
  \frac{d[f(x)]}{d[g(x)]}=\frac{d[f(x)]/dt}{d[g(x)]/dt}
\end{equation}
