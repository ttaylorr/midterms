\chapter{Intermolecular Forces}
\section{Introduction}
The state of matter of a substance is determined by two factors:

\begin{enumerate}
  \item The amount of attractive force between particles
  \item The kinetic energy of the substance
\end{enumerate}

Intermolecular forces are the attractions between adjacent partciles which will
keep molecules close to each other. Temperature, on the other hand, is the
measure of kinetic energy. When the temperature is high, the molecule is
capable of pulling itself apart.

\section{Intermolecular forces}
Several intermolecular interactions are described below. In order of weakest to
strongest, they are the following:
\begin{itemize}
  \item London-dispersion forces
  \item Dipole/induced-dipole interaction
  \item Dipole-dipole interaction
  \item Hydrogen-bonding
\end{itemize}

\subsection{States of Matter}
\begin{description}
  \item[Strong] the intermolecular force that produces solids, unless the
    temperature is particularly high.
  \item[Weak] the intermolecular force that produces gases, unless the
    temperature is particularly low.
\end{description}

Liquids, the second state of matter, represent the tipping point between
intermolecular attraction and freedom due to high kinetic energy.

\subsection{Dipole-dipole interaction}
\begin{description}
  \item[Description] Dipoles orient themselves such that the positive end of one
    molecular dipole aligns with the negative end of another molecular dipole.
  \item[Qualifications] Occurs in a molecule with a permanent dipole. All
    molecules must have molecular dipoles: $\vec{\mu} > 0$.
\end{description}

\subsection{Dipole, induced-dipole}
\begin{description}
  \item[Description] Polar molecules "stretch" the electron cloud of the
    non-polar substance, giving a case of dipole-dipole interaction.
  \item[Qualifications] Occurs in a mixture of polar and non-polar covalent
    molecules.
\end{description}

\subsection{London dispersion forces}
\begin{description}
  \item[Description] A temporary imbalance between atoms or molecules is induced
    because of the movement of electrons, yielding a case of dipole-dipole
    interaction.
  \item[Qualifications] Occurs in all mixtures of non-polar molecules or atoms.
\end{description}

\subsection{Hydrogen bonding}
\begin{description}
  \item[Description] Extreme case of dipole-dipole interaction.
  \item[Qualifications] Must satisfy all qualifications for normal dipole-dipole
    interaction, as well as have one (or more) of the following elements:
    \ce{H2}, \ce{N2}, \ce{O2}, or \ce{F2}.
\end{description}

\section{Properties of \ce{H2O}}
\subsection{Surface tension}
Produced by a net-inward force acting on the surface of a liquid. All particles
on the interior of a liquid are pulled equally toward each other. The molecules
on top are only pulled from the sides and below, which means that the net-force
on the liquid is inwards.

\subsection{Cohesion v. Adhesion}
\begin{description}
  \item[Cohesion] \ce{H2O} sticks to itself.
  \item[Adhesion] \ce{H2O} sticks to other things.
\end{description}

\subsection{Capillary action}
Occurs when the adhesive forces between a liquid and a neighboring surface are
strong, so the liquid pulls itself along the surface. It is a combination of
both cohesion and adhesion.
