\chapter{Percent Composition, Reactions in Solution}

\section{Introduction}
\subsection{Topics}
\begin{enumerate}
  \item Formulas from percent composition
  \item Finding formulas from mass data
  \item Behavior of ionic compounds in water
  \item Percipitation reactions
  \item Acid/base reactions
  \item Gas-forming reactions
  \item Oxidation-reduction reacftions
\end{enumerate}

\section{Emperical Formulae}
\subsection{Description of Substances}
\begin{description}
  \item[Chemical formula] details the chemicals contained in a compound by their
    whole-number ratios. Requires a pure-compound.
  \item[Percent composition] details the elements in a compound by their
    percent-mass makeup of presence in that compound. Works for both mixtures
    and pure compounds.
\end{description}

\subsection{Percent Composition}
To calculate the percent (by mass) of a compound in some substance given a
chemical formula, assume the following process:

\begin{enumerate}
\item Assume 1.00mol of substance is present
\item Determine the molar mass of the substance
\item $\text{\% of x}=(100)\frac{\text{g of x in 1.00mol}}{\text{total mass of 1.00mol}}$
\end{enumerate}

\subsection{Emperical Formula}
If you are given the mass values or percent composition of an unknown substance,
it is possible to determine the empirical formula for that pure compound.

The emperical formula is the simplest whole number ration of chemicals in a
compound. It does give information about the ratios of substances in a compound,
but makes no guarentees about the actual amount in that compound. For example,
the emperical formula of \ce{N2O4} is \ce{NO2}.

\subsubsection{Calculating}
To calculate emperical formulae, follow this process:
\begin{enumerate}
  \item If given a percent composition by mass, assume \SI{100}{\gram} sample.
    Convert percentages into grams.
  \item Convert the mass amount in grams to mols.
  \item Simplify the ratio by dividing by the smallest molar amount. If
    necessary, multiply to reach a whole-number ratio.
\end{enumerate}

\subsection{Hydrates}
Hydrates are ionic compounds which trap water between their crystalline
structure. The amount of water is trapped in a well-defined whole-number ratio.
To analyze a compound of such type, the solid is usually heated until the water
is boiled off, and the anhydrous salt is massed.

To calculate the formula of a hydrate, use the following process:
\begin{enumerate}
  \item Determine the mass of the hydrate, anhydrous solid, and the mass of
    water lost (the difference of the two).
  \item Convert all mass information to mols.
  \item Obtain the ratio between the anhydrous solid and the water.
\end{enumerate}

\section{Molecular Formulae}
In order to determine a molecular formula, additional information about the
compound must be known or determined.

\subsection{Mass Spectrometry}
\begin{itemize}
  \item Make substance a cation.
  \item Shoot particles of that substance past a magnet.
  \item Heavier ions curve less, mass information and identity can be
    determined.
\end{itemize}

\subsection{Conversion}
If both the molecular mass is known (and the empirical mass can be calculated),
then a conversion factor can be obtained.

$$n=\frac{\text{molecular mass}}{\text{empirical mass}}$$

\section{Reactions in Solution}
\subsection{Percipitation Reactions}
To know whether or not a percipitate will form, we must break the reacting
substance into ions, see which can disolve, and determine what remains.

To determine what ions are formed, we can classify all ionic compounds into one
of three categories:

\begin{description}
  \item[Strong electrolyte] Dissociates completly in water\\
    Example: \ce{NaCl(s) -> Na+(aq) + Cl-(aq)}
  \item[Weak electrolyte] Dissociates partially in water\\
    Example: \ce{CH3COOH(aq) <--> H+(aq) + CH3COO-(aq)}
  \item[Non-electrolyte] No ion formation upon dissolution\\
    Example: \ce{C6H12O6(s) -> C6H12O6(aq)}
\end{description}

A table of soluability guidelines follows:

\begin{table}[]
\centering
\caption{Soluable compounds}
\begin{tabular}{|l|l}
\cline{1-1}
\textbf{Soluble Compounds}                               &                                                                                         \\ \cline{1-1}
Almost all salts of \ce{Na+}, \ce{K+}, \ce{NH4+}         &                                                                                         \\ \hline
Salts of \ce{NO3-}, \ce{ClO3-}, \ce{ClO4-}, \ce{CH3CO2-} & \multicolumn{1}{l|}{\textbf{Exceptions}}                                                \\ \hline
Salts of \ce{Cl-}, \ce{Br-}, \ce{I-}                     & \multicolumn{1}{l|}{Halides of \ce{Ag+}, \ce{Hg2+}, \ce{Pb2+}}                          \\ \hline
Salts containing \ce{F-}                                 & \multicolumn{1}{l|}{Flourides of \ce{Mg2+}, \ce{Ca2+}, \ce{Sr2+}, \ce{Ba2+}, \ce{Pb2+}} \\ \hline
Salts of \ce{SO42-}                                      & \multicolumn{1}{l|}{Sulfates of \ce{Ca2+}, \ce{Sr2+}, \ce{Ba2+}, \ce{Pb2+}, \ce{Ag+}}   \\ \hline
\end{tabular}
\end{table}

\begin{table}[]
\centering
\caption{Insoluble compounds}
\begin{tabular}{|l|l|}
\hline
\textbf{Insoluble Compounds}                                               & \textbf{Exceptions}                                 \\ \hline
\begin{tabular}[c]{@{}l@{}}Salts of \ce{CO3^2-}, \ce{PO4^3-}, \ce{C2O4^2-},\\ \ce{CrO4^2-}, \ce{S^2-}\end{tabular} & Salts of \ce{NH4+}, alkali metal cations            \\ \hline
Metal hydroxides and oxides                                                & Alkali metal hydroxides, \ce{Ba(OH)2}, \ce{Sr(OH)2} \\ \hline
\end{tabular}
\end{table}

\subsubsection{Net Ionic Equation}
To calculate a percipitation, a net-ionic equation must first be obtained.

All pure solids, liquids and gases will remain intact molecules. Aqueous
substances will break apart into their ionic components. An example is shown
below:

\ce{K2CrO4(aq) + Pb(NO3)2(aq) -> 2KNO3(aq) + PbCrO4(s)}\\

Has an ionic equation of...\\

\ce{2K+(aq) + CrO4^2-(aq) + Pb^2+(aq) + 2NO3^-(aq) -> \\ 2K+(aq) + 2NO3^-(aq) +
PBCrO4(s)}\\

Which simplifies to...\\

\ce{CrO4^2-(aq) + Pb^2+(aq) -> PBCrO4(s)}

\subsection{Acid/Base Reactions}
\begin{description}
  \item[Acid] a substance which will increase the amount of \ce{H+} ions in
    \ce{H2O}.
  \item[Base] a substance which will increase the amount of \ce{OH-} ions in
    \ce{H2O}.
\end{description}

A table of acids and bases are given below:

\begin{table}[]
\centering
\caption{Strong acids and bases}
\begin{tabular}{|l|l|l|l|}
\hline
\multicolumn{2}{|l|}{Acids} & \multicolumn{2}{l|}{Bases}    \\ \hline
\ce{HCl}    & Hydrochloric acid  & \ce{LiOH}    & Lithium hydroxide   \\ \hline
\ce{HBr}    & Hydrobromic acid   & \ce{NaOH}    & Sodium hydroxide    \\ \hline
\ce{HI}     & Hydroiodic acid    & \ce{KOH}     & Potassium hydroxide \\ \hline
\ce{HNO3}   & Nitric acid        & \ce{Ba(OH)2} & Barium hydroxide    \\ \hline
\ce{HClO4}  & Perchloric acid    & \ce{Sr(OH)2} & Strontium hydroxide \\ \hline
\ce{H2SO4}  & Sulfuric acid      &              &                     \\ \hline
\end{tabular}
\end{table}

\begin{table}[]
\centering
\caption{Weak acids and bases}
\begin{tabular}{|l|l|l|l|}
\hline
\multicolumn{2}{|l|}{Acids}  & \multicolumn{2}{l|}{Bases} \\ \hline
\ce{HF}       & Hydrofluoric acid & \ce{NH3}   & Ammonia       \\ \hline
\ce{H3PO4}    & Phosphoric acid   &            &               \\ \hline
\ce{H2CO3}    & Carbonic acid     &            &               \\ \hline
\ce{CH3COOH}  & Acetic acid       &            &               \\ \hline
\ce{H2C2O4}   & Oxalic acid       &            &               \\ \hline
\ce{H2C6H4O6} & Tartaric acid     &            &               \\ \hline
\ce{H3C6H5O7} & Citric acid       &            &               \\ \hline
\ce{HC9H7O4}  & Aspirin           &            &               \\ \hline
\end{tabular}
\end{table}

When acids and bases react together, they yield a salt and water. A salt is an
ionic compound which is usually soluable, and composed of the "key" of the base
and the rest of the acid.

An example is shown below:\\

\ce{CH3COOH(aq) + KOH(aq) -> KCH3COO(aq) + H2O(l)}\\
\ce{CH3COOH(aq) + OH-(aq) -> CH3COOH(aq) + H2O(l)}

\subsection{Gas-Forming Reactions}
Gas forming reactions may be classified as a member of a set of four reactions
that takes on a particular pattern. They are all listed below:
\begin{enumerate}
  \item Metal-carbonate or hydrogen-carbonate + acid \ce{->} metal salt +
    \ce{CO2(g) + H2O(l)}
  \item Metal-sulfide + acid \ce{->} metal salt + \ce{H2S(g)}
  \item Metal-sulfite + acid \ce{->} metal salt + \ce{SO2(g) + H2O(l)}
  \item Ammonium salt + strong base \ce{->} metal salt + \ce{NH3(g) + H2O(l)}
\end{enumerate}

\subsection{Redox reactions}
For this exam, a limited definition of a redox reaction will suffice. A redox
reaction is one wherein a element is formed or consumed, or the charge of a
substance changes.

To assign an oxidation number to a reaction, the following processes may be
implemented.

\begin{description}
  \item[Pure elements] 0
  \item[Monatomic ions] Charge of the ion
  \item[\ce{F} in compounds] -1
  \item[Halogens] -1 unless with \ce{O} or \ce{F}
  \item[\ce{H} and \ce{O}] +1 for \ce{H}, -2 for \ce{O}
  \item[Otherwise] Work backwards
\end{description}
