\chapter{Gas Laws}
\section{Introduction}
\subsection{Topics}
\begin{enumerate}
  \item Properties
  \item Pressure: derivation of, units
  \item Boyle's, Charles's, Gay-Lussac's, Avogadro's laws
  \item Combined, Ideal, Dalton, graham, van der Waals's Laws
  \item Kinetic molecular theory
  \item Stoichiometry
\end{enumerate}

\section{Properties}
Gas has many unique properties not observable in solids and liquids. Because of
these extra properties, it is useful to be able to talk more in detail about the
properties of gas.

\begin{description}
  \item[Macroscale] Pressure, volume, temperature, number of particles
  \item[Microscale] Intermolecular forces, velocity, kinetic energy, size of
    particles
\end{description}

\subsection{Pressure}
Pressure deals with the force of an impact over the surface area impacted. Gas
molecules are constantly in motion and continuously collide with the walls of
the container they are placed in. These collisions impart a force on the
container, and thus pressure is created.

Pressure is measured in $atm$, $bar$, $pa$, $mmHg$, $torr$, and $psi$. We
usually use $atm$, $pa$, $mmHg$, and $torr$.

\subsubsection{Measuring Pressure}
\begin{enumerate}
  \item A glass tube containing a vacuum is placed atop a bowl of liquid, such
    as \ce{Hg}.
  \item Liquid is pushed up the tube.
  \item The higher the air pressure, the more the liquid moves.
\end{enumerate}

\section{Gas Laws}
\subsection{Boyle's Law}
\begin{description}
  \item[Description] Boyle's law relates the amount of pressure and volume in
    two containers of gasses.
  \item[Constants] Amount of gas particles, temperature.
  \item[Units] Pressure and volume have to be consistent units, but can be
    anything as long as it makes sense.
\end{description}

\begin{equation}
  PV=k
\end{equation}

\begin{equation}
  P_1V_1=P_2V_2
\end{equation}

\subsection{Charles's Law}
\begin{description}
  \item[Description] Charles's law describes the relationship between
    temperature and volume.
  \item[Constants] Pressure and amount of gas present.
  \item[Units] Volume must be self-consistent between the two sides of the
    equation. Temperature must be on an absolute scale, such as Kelvin.
\end{description}

\begin{equation}
  \frac{V}{T}=k
\end{equation}

\begin{equation}
  \frac{V_1}{T_1}=\frac{V_2}{T_2}
\end{equation}

\subsection{Combined Gas Law}
\begin{description}
  \item[Description] Relates the amount of pressure, volume, and temperature.
  \item[Constants] Amount of gas present.
  \item[Units] Pressure and volume must be self consistent. Temperature must be
    on an absolute scale, such as Kelvin.
\end{description}

\begin{equation}
  \frac{P_1V_1}{T_1}=\frac{P_2V_2}{T_2}
\end{equation}

\subsection{Avagadro's Law}
\begin{description}
  \item[Description] Said that only the number of gas molecules is important in
    determining volume.
  \item[Constants] Pressure and temperature of the gas.
  \item[Units] Volume must be self consistent, $n$ must be in $mol$s.
\end{description}

\begin{equation}
  \frac{V_1}{n_1}=\frac{V_2}{n_2}
\end{equation}


\subsection{Ideal Gas Law}
\begin{description}
  \item[Description] Based on theories put forth by Avagadro that volume and
    number of particles are proportional. Combines all other gass laws.
  \item[Constants] None.
  \item[Units] Pressure and volume may be any self-consistent units. Temperature
    must be on an absolute scale, such as Kelvin. $n$ must be in $mol$s.
\end{description}

\begin{equation}
  PV=nRT
\end{equation}

\begin{equation}
\begin{split}
  PV & =nRT\\
  PV & =\frac{m}{M}RT\\
  PM & =\frac{m}{V}RT\\
  & =DRT\\
  D & =\frac{PM}{RT}
\end{split}
\end{equation}

\section{Stoichionetry}

Avogadro's law can be applied to stoichiometry so that the mole ratio in a gas
equation can also describe the volume ratio of gases in an equation. This is the
Law of Combining Volumes.

An example of the above equations is shown below:

\textit{Find the volume of hydrogen needed to react completely with
\SI{17.5}{\liter} of \ce{N2(g)} at \SI{300.0}{\kelvin} and 1.25atm.}

\ce{N2(g) + 3H2(g) -> 2NH3(g)}

\begin{equation}
\begin{split}
  n_{\ce{N2}} &= V\left(\frac{P}{RT}\right)\\
  V_{\ce{H2}} &= n\left(\frac{RT}{P}\right)\\
\end{split}
\end{equation}

\begin{equation}
\begin{split}
  V_{\ce{H2}} &= 3n_{\ce{N2}}\left(\frac{RT}{P}\right)\\
  &= 3V_{\ce{N2}}\left(\frac{P}{RT}\right)\left(\frac{RT}{P}\right)\\
  &= 3V_{\ce{N2}}
\end{split}
\end{equation}

\subsection{Standard Temperature and Pressure}
If temperature is at $273.15K$ and pressure is equal to $1.00atm$, then $1mol$
of gas is $22.41L$ of volume.

\section{Mixtures}
\subsection{Dalton's Law (Partial Pressures)}
If a mixture is composed of all gases, and the pressure of said mixture is equal
to $P$, then the following can be stated:

\begin{equation}
\begin{split}
  P_{total} &= \sum_{i=1}^{n} P_i\\
  &= P_1 + P_2 + P_3 + \ldots + P_n
\end{split}
\end{equation}

Since molar fractions apply here, we can calculate the mol. fraction of a given
substance to be:

\begin{equation}
  X_a=\frac{\text{mol}_A}{\text{mol}_{total}}
\end{equation}

and use that relationship to obtain the following:

\begin{equation}
  P_a=X_a\sum_{i=1}^{n}P_i
\end{equation}

\begin{equation}
  P_{total} = \sum_{i=1}^{n} X_iP_t
\end{equation}

\section{Kinetic Molecular Theory}
\begin{enumerate}
  \item Gas molecules are far apart from each other, s.t. gases are mostly empty
    space.
  \item Gas molecules have continuous, rapid motion. They collide elastically
    with each other and the walls of the container.
  \item Kinetic energy is proportional to temperature. All gases at the same
    temperature have the same kinetic energy, regardless of their mass.
\end{enumerate}

\subsection{Root-mean-square Speed}
In a sample of lots of gas particles, a range of energies may be found. To
determine the average speed of the particles over the entire sample, the
root-mean-squared speed is calculated.

Here, $R$ represents the gas constant, $T$ is temperature in Kelvin, and $M$ is
the molar mass in $kg/mol$.

\begin{equation}
  v_{rms} = \sqrt{\frac{3RT}{M}}
\end{equation}

\subsection{Movement}
\begin{description}
  \item[Diffusion] mixing of gas molecules
  \item[Effusion] movement of gas from one container through a
    (atomically-sized) hole.
\end{description}

\subsubsection{Graham's Law}
\begin{equation}
  \frac{\text{rate}_a}{\text{rate}_b} = \sqrt{\frac{M_b}{M_a}}
\end{equation}

\section{Non-ideal gases}
Gases begin to deviate from the ideal behavior defined above in a few different
conditions:

\begin{enumerate}
  \item High pressure and/or small volumes. In this case, it is no longer
    correct to assume that gases are mostly empty space.
  \item At low temperatures, the attractive forces between gas particles becomes
    noticable, and they become "sticky".
\end{enumerate}

\subsection{van der Waals equation}
The van der Waals equation is essentially the Ideal Gas Law with a few
correction factors thrown in. These factors are described below:

\begin{description}
  \item[$a$] the intermolecular forces between gas particles
  \item[$b$] the volume of the gas molecules themselves
\end{description}

These values will always be given in a problem.

\begin{equation}
  \left(P-a\left[\frac{n}{V}\right]^2\right)\left(V-nb\right) = nRT
\end{equation}
