\chapter{Thermodynamics}
\section{Introduction}
\subsection{Topics}
\begin{enumerate}
  \item Forms of energy
  \item Heat and temperature
  \item Heat transfer
  \item Work
  \item Calorimetry
  \item Heats of reaction
  \item Hess's Law
  \item Standard enthalpy of formation
  \item Entropy
  \item Gibbs free energy
\end{enumerate}

\section{Forms of Energy}
\begin{description}
  \item[Kinetic Energy] energy of motion of a particular object
    \begin{description}
      \item[Thermal] energy of molecular motion
      \item[Mechanical energy] energy of macroscopic objects in motion
      \item[Electrical] motion of electrons
    \end{description}
  \item[Potential Energy] energy of the position of a particular object
    \begin{description}
      \item[Chemical] energy stored in bonds
      \item[Gravitational] attraction due to position relative to another
        object
      \item[Electrostatic] Couloumb's law
    \end{description}
\end{description}

\section{Temperature and Heat}
\begin{description}
  \item[Heat] the total thermal energy content of a sample. This depends on the
    amount of motion and the number of particles present.
  \item[Temperature] Teh average kinetic energy of the particles in a sample.
    This amount is independent of the sample size.
\end{description}

\subsection{Heat Transfer}
Heat is always transferred from hot objects to cold objects. This occurs because
collisions between molecules can transfer energy from one energy to another.

Equilibrium is reached when the two objects have the same average kinetic
energy.

\begin{description}
  \item[Exothermic] Heat is transferred from a system to the surroundings.
  \item[Endothermic] Heat is transferred from the surroundings to the system.
\end{description}

\subsubsection{Energy Units}
\begin{description}
  \item[Joule] The standard metric for energy and work
  \item[cal] 1 cal will increase the temperature of \SI{1}{\gram} of \ce{H2O} by
    \SI{1}{\degreeCelsius}
\end{description}

\subsection{Energy transfer}
\begin{description}
  \item[Description] Equation to calculate the change in energy of a
    non-reactive system undergoing no phase-change(s).
  \item[Units] $q$ is in \SI{}{\joule}, $m$ is in mass or moles, $C$ is specific
    heat in mass or moles, consistent with the units of $m$. $t$ is the
    temperature change.
\end{description}
\begin{equation}
  q=mc\Delta T
\end{equation}

\begin{description}
  \item[Description] Equation to calculate the energy change of a non-reactive
    system at a phase change.
  \item[Units] $q$ is in \SI{}{\joule}, $m$ \SI{}{\mol}, $\Delta H$ is a heat of
    fusion.
\end{description}
\begin{equation}
  q=n\Delta H
\end{equation}

\begin{description}
  \item[Description] Law of conservation of energy.
\end{description}
\begin{equation}
  \sum_{i=1}^{n} q_i = 0
\end{equation}

\subsection{Other forms of energy}
In a piston, for example, work is converted from natural processes like the
expansion of gas into useful work. The equation governing this action is
displayed below.

\begin{equation}
  \Delta U = q + w
\end{equation}

Work causes some macroscopic motion against a resistance. The work referred to
in the above example is pressure-volume work, because the "work" is the
expansion of gas. These types of processes are goverened by:

\begin{equation}
  W = -P\Delta V
\end{equation}

\begin{table}[]
\centering
\begin{tabular}{lll}
Heat transferred to system from surroundings & $q > 0$ & $\Delta E > 0$ \\
Heat transferred from system to surroundings & $q < 0$ & $\Delta E < 0$ \\
Work done on system by surroundings          & $w > 0$ & $\Delta E > 0$ \\
Work done by system on surroundings          & $w < 0$ & $\Delta E < 0$
\end{tabular}
\end{table}

\section{Calorimetry}
Calorimetry is an experimental procedure used to measure energy changes.
Experiments are preformed either at constant pressure or at constnat volume.
Constant volume is referred to as bomb calorimetry. Constant pressure is
referred to as "coffee cup" calorimetry.

\begin{equation}
  q_{rxn} + q_{bomb} + q_{\ce{H2O}} = 0
\end{equation}

$q_{bomb}$ refers to the energy released in the reaction that heats the
calorimeter itself. The calorimeter has the same $\Delta T$ as the surrounding
water. The value of $m$ is not included in this calculation because the value of
$C_{bomb}$ already takes it into account.

\begin{equation}
  q_{bomb} = C_{bomb}\Delta T
\end{equation}

\section{Hess's Law}
Hess's Law is used to calculate the enthalpy of a reaction when subcomponents of
the reaction's enthalpy are known, but the original equation's enthalpy is not.
It takes advantage that enthalpy, being an energy, is a state function. An
example is shown below:

\textit{ex. \ce{N2} and \ce{O2} combined to form \ce{N2O5}.}\\
\ce{N2 + 2O2 -> N2O4} \SI{800}{\kilo\joule\per\mol}\\\\

\ce{N2 + O2 -> N2O2} \SI{-200}{\kilo\joule\per\mol} released\\
\ce{N2O2 + O2 -> N2O4} \SI{1000}{\kilo\joule\per\mol} released\\
\SI{800}{\kilo\joule\per\mol} released

\subsection{Equation Manipulations}
\begin{enumerate}
  \item The coefficients of a reaction may be changed by multiplying the entire
    equation by a scalar.
  \item However, if the scalar is negative, the reaction must first be reversed
    such that the products become the reactants, and vice-versa.
\end{enumerate}

\section{Standard Heats of Formation}
\begin{equation}
  \Delta H^{\circ}_{rxn} = \sum_{i=1}^{n} n_i\Delta H^{\circ}_{\text{products}} -
  \sum_{i=1}^{n} n_i\Delta H^{\circ}_{\text{reactants}}
\end{equation}

\section{Spontaneous Reactions}
Spontaneous reactions occur without outside intervention. It does not detail how
fast a reaction will occur, or how much product will be formed, but it tells you
whether or not a reaction will occur.

\begin{description}
  \item[Product-favored] reactions that contain mostly products at the end
    (therefore having a high percent-yield).
  \item[Reactant-favored] reactions that contain mostly reactants at the end
    (therefore having a low percent-yield).
\end{description}

Two factors will determine whether or not a reaction is spontaneous.
\begin{enumerate}
  \item Heat. Processes which release heat are more likely to be spontaneous.
    This is because these reactions do not need an outside source of energy.
    Reactants are tapping into their own stored energy to carry out the
    reaction. \textit{N.B.: this is not the only requirement for a reaction to
    be spontaneous}.
  \item Entropy. Entropy is the dispersal of matter and energy in a system.
\end{enumerate}

\section{Second Law of Thermodynamics}
Spontanous proccesses increase the entropy, or disorder, of the universe.

\begin{equation}
\Delta S_{\text{universe}} \geq 0
\end{equation}

\section{Third Law of Thermodynamics}
The entropy of a perfect crystalline structure will be $0$. Anything not at zero
Kelvin will have a positive value of entropy.

\begin{equation}
  \Delta S = \frac{q}{T}
\end{equation}

The change in entropy for a spontaneous process may be positive or negative, but
it is more likely to be positive if the process is spontaneous.

\section{Examples of Entropy Change}
\begin{description}
  \item[Change of state] Gases have far greater entropy values than liquids,
    which are somewhat greater than solids.
  \item[Change of temperature] Objects at low temperature have lower entropy
    than objects at high temperature.
  \item[Mixing substances] Pure substances have lower entropy than substances
    that are mixtures.
  \item[Molecules] Fewer molecules have less entropy than many molecules.
\end{description}

\begin{equation}
  \Delta S^{\circ}_{rxn} = \sum_{i=1}^{n} n_i\Delta S^{\circ}_{\text{products}} -
  \sum_{i=1}^{n} n_i\Delta S^{\circ}_{\text{reactants}}
\end{equation}

\begin{equation}
  \begin{aligned}
    \Delta S^{\circ}_{\text{surroundings}} &= \frac{q_{\text{surroundings}}}{T}\\
    &= \frac{-\Delta H^{\circ}_{\text{system}}}{T}
  \end{aligned}
\end{equation}

\section{Gibbs Free Energy}
Also referred to as Helmmholtz Free Energy, this value takes into consideration
both enthalpy and entropy changes to determine whether or not a reaction will be
spontaneous. The "free" energy is the energy that is available to do work.

The sign on $\Delta G$ tells us whether or not the reaction will be spontaneous
(positive values are, negative ones aren't).

\begin{equation}\begin{aligned}
  \Delta G &= \sum_{i=1}^{n} n_i\Delta G^{\circ}_{\text{products}} -
              \sum_{i=1}^{n} n_i\Delta G^{\circ}_{\text{reactants}}\\
           &= \Delta H - T\Delta S
\end{aligned}\end{equation}

\begin{table}[]
\centering
\begin{tabular}{l|l|l|}
\cline{2-3}
 & \textbf{\begin{tabular}[c]{@{}l@{}}$\Delta S > 0$\\ Disorder favored\end{tabular}} & \textbf{\begin{tabular}[c]{@{}l@{}}$\Delta S < 0$\\ Order not favored\end{tabular}} \\ \hline
\multicolumn{1}{|l|}{\textbf{\begin{tabular}[c]{@{}l@{}}$\Delta H > 0$\\ Endothermic disfavored\end{tabular}}} & \begin{tabular}[c]{@{}l@{}}$\Delta G < 0$ if $T\Delta S > \Delta H$\\ Process is entropically driven.\end{tabular} & $\Delta G > 0$ \\ \hline
\multicolumn{1}{|l|}{\textbf{\begin{tabular}[c]{@{}l@{}}$\Delta H < 0$\\ Exothermic favorable\end{tabular}}} & $\Delta G < 0$ & \begin{tabular}[c]{@{}l@{}}$\Delta G > 0$ when $\left|\Delta H\right| > \left|T\Delta S\right|$\\ Process is enthalpically driven.\end{tabular} \\ \hline
\end{tabular}
\end{table}
