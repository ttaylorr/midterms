\chapter{Equations and Analysis}
\section{Introduction}
\subsection{Topics}
\begin{enumerate}
  \item Mass relationships
  \item Limiting and excess reagents
  \item Percent yield
  \item Chemical analysis
  \item Concentration and solution stoichiometry
  \item Spetctrophotometry
  \item Oxidation-reduction reactions
\end{enumerate}

\section{Stoichiometry}
Stoichiometry involves the use of a balanced chemical equations to determine
the relationship between similar properties of two seperate substances in an
equation. Properties that can be measured using this technique include mass,
volume, moles, pressure, and concentration.

\subsection{Problem Solving}
To solve a problem using stoichometry, implement the following technique:
\begin{enumerate}
  \item Write a balanced equation
  \item Convert the given information into moles
  \item Use the mole ratio to convert between different substances
  \item Take the mole amount of the unknown and convert into a different unit
\end{enumerate}

An example of this technique in action follows below:

First, a chemical equation is written.\\
\ce{CH4 + 2O2 -> CO2 + 2H2O}\\

We are trying to find the amount of \ce{CO2} produced from \SI{50}{\gram} of
\ce{CH4}.

$$\left(\frac{\SI{50}{\gram} \ce{CH4}}{1}\right)\left(\frac{\SI{1}{\mol}
\ce{CH4}}{\SI{16.04}{\gram}}\right)\left(\frac{\SI{1}{\mol} \ce{CO2}}{\SI{1}{\mol}
\ce{CH4}}\right)\left(\frac{\SI{44.01}{\gram} \ce{CO2}}{\SI{1}{\mol}
\ce{CO2}}\right) = \SI{137}{\gram} \ce{CO2}$$

\section{Limiting Reactants}
Sometimes, an excess of one reactant is present as compared to a limiting
amount of another reactant. For example, if you a grilling hot-dogs and there
are 10 dogs per package, but only 8 buns per package, the buns are the limiting
"reagent."

To solves these types of problems, do the stoichometry both ways and see which
can produce less of the same product. Whichever reactant you started with is the
limiting reagent.

An example is shown below:\\

\textit{\SI{100.}{\gram \ce{Xe}} reacts with \SI{100.}{\gram \ce{F2}} to produce
\ce{XeF6}}\\

\ce{Xe + 3F2 -> XeF6}\\

Starting with \SI{100.}{\gram \ce{Xe}}:
$$\left(\SI{100.}{\gram \ce{Xe}}\right)\left(\frac{\SI{1}{\mol
\ce{Xe}}}{\SI{131.29}{\gram \ce{Xe}}}\right)\left(\frac{\SI{1}{\mol
\ce{XeF6}}}{\SI{1}{\mol \ce{Xe}}}\right)\left(\frac{\SI{245.28}{\gram
\ce{XeF6}}}{\SI{1}{\mol \ce{XeF6}}}\right) = \SI{187}{\gram \ce{XeF6}}$$

Starting with \SI{100.}{\gram \ce{F2}}:
$$\left(\SI{100.}{\gram \ce{F2}}\right)\left(\frac{\SI{1}{\mol
\ce{F2}}}{\SI{38.00}{\gram \ce{F2}}}\right)\left(\frac{\SI{1}{\mol
\ce{XeF6}}}{\SI{3}{\mol \ce{F2}}}\right)\left(\frac{\SI{245.28}{\gram
\ce{XeF6}}}{\SI{1}{\mol \ce{XeF6}}}\right) = \SI{215}{\gram \ce{XeF6}}$$

We can determine that \SI{100.}{\gram \ce{Xe}} is the limiting reagent since it
produces less \ce{XeF6}.

\section{Chemical Analysis}
\subsection{Identifying impurities}

Most impurity problems deal with finding the amount of some substance present in
a solution when we know information about how that solution reacts with yet
another substances. From this, we can obtain two equations, and do stoichometry
between the two in order to determine how much of the original substance is
present in the solution.

An example problem follows:\\

\textit{a \SI{275}{\gram} of impure water is treated with dilute \ce{H2CO3}
  solution to test for the presence of calcium chloride. After filtration,
\SI{0.0034}{\gram \ce{CaCO3}} is collected. Determine the mass percentage of
\ce{CaCl2} in the water sample, assuming all of the calcium present is in the
form of \ce{CaCl2}.}\\

\ce{Ca^2+ + CO3^2- -> CaCO3}\\
\ce{CaCl2 -> Ca^2+ + 2Cl-}\\

Stoichometry is done to convert between \ce{CaCO3} and the mass amount of
\ce{CaCl2}. A mass percent is then obtained using the formula from the previous
chapter.

\subsection{Identifying hydrocarbons}
A hydrocarbon comes in the form \ce{C_xH_y} or \ce{C_xH_yO_z}. Hydrocarbons can
be easily identified using the above processes when undergoing combustion. The
problem-solving technique is outlined below:

\begin{enumerate}
  \item Determine the hydrocarbon.
  \item Combust it, and write a balanced chemical equation (\ce{CO2} and
    \ce{H2O} are always produced).
  \item Use the amount of \ce{CO2} to determine how much \ce{C} the hydrocarbon
    contained.
  \item Use the amount of \ce{H2O} produced to determine how much \ce{H} is
.    contained in the hydrocarbon.
  \item Use the mass of the hydrocarbon subtracting away the mass of the \ce{C}
    and \ce{H} to determine the mass of \ce{O} left over.
\end{enumerate}

An example is shown below:

\textit{ex. \SI{14.75}{\gram} of a hydrocarbon with the formula \ce{C_xH_yO_z}
  is combusted. After combustion, \SI{24.1981}{\gram} \ce{CO2} and \SI{9.9055}{\gram} \ce
{H2O} remain. Determine the emperical formula of the hydrocarbon.}

Carbon analysis: \ce{CO2 -> C + 2O}

$$(\SI{24.1981}{\gram \ce{CO2}})\left(\frac{\SI{1}{\mol
\ce{CO2}}}{\SI{44.04}{\gram \ce{CO2}}}\right)\left(\frac{\SI{1}{\mol
\ce{C}}}{\SI{1}{\mol \ce{CO2}}}\right) = \SI{.550}{\mol \ce{C}}$$

Hydrogen analysis: \ce{H2O -> 2H + O}

$$(\SI{9.9055}{\gram \ce{H2O}})\left(\frac{\SI{1}{\mol
\ce{H2O}}}{\SI{18.01}{\gram \ce{H2O}}}\right)\left(\frac{\SI{2}{\mol
\ce{H}}}{\SI{1}{\mol \ce{H2O}}}\right) = \SI{1.10}{\mol \ce{H}}$$

Oxygen analysis:

$$\SI{14.75}{\gram} - (\SI{6.60}{\gram \ce{C}} + \SI{1.11}{\gram \ce{H}}) =
\SI{7.04}{\gram \ce{O}}$$

Then use the tactic outlined in the previous chapter to determine the emperical
formula by simplifying the molar ratio obtained above.

Following these steps, the unknown hydrocarbon is: \ce{C5H10O4}.

\section{Percent Yield}
\begin{equation}
  \text{percent yield} = \frac{\text{actual yield}}{\text{theoretical
  yield}}(100)
\end{equation}

\begin{description}
  \item[Theoretical yield] the amount of product obtained by stoichometric
    ratios.
  \item[Actual yield] the amount of product collected from an experiment.
\end{description}

This is not the same as percent error.
