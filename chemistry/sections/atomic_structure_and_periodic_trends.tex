\chapter{Atomic Structure and Periodic Trends}
\section{Electron-Configuration Notation}
A system of numbers and letters is used to designate electron configuration.
For example: \ce{1s^2 2s^2 2p^6 3s^2 3p^2}.

\begin{enumerate}
  \item the level number is used
  \item the level designation for the shape of the orbital
  \item a superscript is used to represent the number of electrons in that
    specific orbital.
\end{enumerate}

\subsection{Noble-gas Notation}
To simplify writing out all of these numbers, you can include the noble-gas
most-closely following behind of the element you are trying to describe, and
then only describe the differing electrons using the notation above.  For
example: \ce{Ne 3s^2 3p^2}.

\begin{description}
  \item[Highest occupied level] the electron containing main energey level with
    the highest quantum number
  \item[Inner-shell electrons] the electrons that are contained in levels with a
    lower quantum state
  \item[Noble-gas configuration] an outer main-energey level that is fully
    occupied, usually by eight electrons.
\end{description}

\section{Periodic Table and Trends}
\subsection{The Periodic Table}
\subsubsection{Mendeleeve's Table}
In 1860, the first Internation Conference of Chemists was held and at that time,
the first periodic table was discussed.  In 1869, 9 years later, that table was
published in a textbook for college students.

Mendeleev was the first to generate a periodic table.  He used atomic mass as
the basis for his table, arranging elements by increasing mass.  Some elements,
however, were out of place when compared against the periodic tables of today.
He was able to, despite these faults, see periodic trends, which allowed him to
predict with a high-degree of accuracy, the qualities of yet undiscovered
elements.

\subsubsection{Improvements}
Moseley observed that elements were better fit as ordered by their nuclear
charge, not atomic mass.  The nuclear charge is due to the number of protons,
which led to the usage of atomic number for description.

\textbf{Periodic Law} states that the physical and chemical properties of the
elements are periodic functions of their atomic numbers.

\subsection{Modern-day}
Today, the \textit{Periodic Table} is just an arrangement of elements in the
order of their atomic numbers, such that elements of similar properties fall in
the samily column or group.  However, there are but a few differences from
Mendeleeve's original table:

\begin{enumerate}
  \item There are more elements now as compared to the first introduction of the
    periodic table
  \item The discovery of Noble Gases and the synthesis of the Lanthanides and
    Actinides
  \item The general arangement has changed over time as well
\end{enumerate}

\section{Periodic Trends}
\subsection{Atomic Radii}
Atomic radius is defined as one half of the distance between the nuclei of
identical atoms that are bonded together.  Atomic radii decreases as we move
across a period (increase in positive charge) and increases as we descend a
family (increasing the main energy level).

\subsection{Ionization Energy}
Ionization energy is the energy required to remove one electron from a neutral
atom of an element.  It is defined by the following chemical equation: \ce{A +
energy -> A^+ + e^-}.  Ionization energy increases as you move right across a
period, and decreases as you move down a family.

\subsubsection{Multiple Ionization Energies}
There are multiple ionization energies, each which involves removing another
electron from the atom.  These are referred to as $IE_2$, $IE_3$, and so on.

\subsection{Electron Affinity}
Electron affinity is the amount of energy change that results when an electron
is acquired by a neutral atom.

\subsection{Ionic Radii}
This trend represents the radius of an ion of an element.  Cations have ionic
radii that are smaller than the neutral atom, where as anions have ionic radii
that are typically larger than the neutral atom.  This happens because of the
addition and removal of electrons in the outer shell.

\subsection{Electronegativity}
Electronegativity is the measure of the ability of an atom in a chemical
compound to attract electrons.
