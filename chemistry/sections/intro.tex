\chapter{Introduction}

\section{Types of Chemistry}
\begin{description}
  \item[Inorganic Chemistry] relates to the chemistry of molecules that do not
    contain a carbon atom
  \item[Organic Chemistry] the chemistry of molecules which do contain a carbon
    atom.  These molecules may also contain a hydrogen atom, and always contain
    carbon bonds.
  \item[Biochemistry] the chemistry of proteins and large bio-molecules
  \item[Analytical] evolving methods to identify and measure things (relating to
    chemistry)
  \item[Theoretical] considered the \textit{opposite} of analytical chemistry.
    Theoretical chemistry is focused on trying to preduct things about chemistry
    using calculus.
  \item[Physical] uses the theory developed from the above discipline to compare
    it against what \textit{actually happened}
\end{description}

\section{Types of Matter}
\begin{figure}[h]
  \Tree [.Matter [.{Pure Substances} Elements Compounds ] [.{Mixtures}
Heterogenous Homogenous ] ]
  \caption{The types of matter.}
\end{figure}

For each type of matter, a definition and several real-world examples are found
below.

\begin{description}
  \item[Elements] matter that cannot be broken down into more pure substances
    \textit{i.e., they have but only one unique type of atom}
    \begin{itemize}
      \item Carbon
      \item Hydrogen
      \item Oxygen
    \end{itemize}
  \item[Compounds] pure substances made ov several elements of which are
    \textit{chemically} joined together
    \begin{itemize}
      \item \ce{H2O}
      \item \ce{CO2}
    \end{itemize}
  \item[Mixtures] impure combinations of two or more substances and are
    \textit{not} chemically joined
    \begin{description}
      \item[Homogenous] uniform content throughout
        \begin{itemize}
          \item Alloys of metal
          \item Filtered air
        \end{itemize}
      \item[Heterogenous] non-uniform content throughout
        \begin{itemize}
          \item Non-pasturized apple juice
        \end{itemize}
    \end{description}
\end{description}

\section{Separation of a Mixture}
\begin{description}
  \item[Filtration] used on a heterogenous mixture by means of a porris barrier.
    Exploits the physical size difference \textit{ex., using a coffee filter to
    make coffee}.
  \item[Distillation] used on homogenous mixtures and takes advantange of
    temperature differences. \textit{ex., alcohol boils at temperature A, water
      boils at temperature B.  A temperature between the two is picked to boil
    off one compound, and then condense it later.}
  \item[Chromatography] used on both types of mixtures to determine properties
    of compounds to seperate.  In practice, a mixture is placed on the end of a
    piece of paper, and water draws out all the compounds contained within it.
\end{description}

\section{General Properties of Substances}
\begin{description}
  \item[Intensive] properties of substances which are not related to the amount
    of that substance which is present
    \begin{itemize}
      \item Boiling point
      \item Specific heat
      \item Density
    \end{itemize}
  \item[Extensive] properties of substances which are functions of the amount of
    that substance which is present
    \begin{itemize}
      \item Mass
      \item Volume
      \item Quantity of atoms
    \end{itemize}
\end{description}

\subsection{Chemical vs. Physical}
\begin{description}
  \item[Chemical] observed when something changes into a new substance
    \begin{itemize}
      \item Bond
      \item Toxicity
      \item Reactivity
      \item Flamability
    \end{itemize}
  \item[Physical] observed without changing chemical properties
    \begin{itemize}
      \item Color
      \item Density
      \item Temperature
    \end{itemize}
\end{description}

\section{SI Units}
The SI-system is the unit of measurements most widely used by scientists.  It
defines a prefix-base system which is useful in representing both large and
small numbers.

\subsection{Base Units}
\begin{description}
  \item[Length] meter \textit{m}
  \item[Mass] kilogram \textit{kg}
  \item[Time] second \textit{sec}
  \item[Teperature] kelvin \textit{k}
  \item[Amount of Substance] mole \textit{mol}
\end{description}

\subsection{Derived Units}
\begin{description}
  \item[Area] \si{m^2}
  \item[Volume] \si{m^3}
  \item[Density] \si{kg/m^3}
  \item[Molar Mass] \si{kg/mol}
  \item[Concentration] \si{mol/L}
  \item[Molar Volume] \si{L/mol}
  \item[Velocity] \si{m/s}
  \item[Force] \si{m/s^2}
\end{description}

\subsection{Prefix System}
\begin{table}[h]
\begin{tabular}{@{}lll@{}}
\toprule
        & Symbol  & Modifier \\ \midrule
  mega  & M       & \si{10^6} \\
  kilo  & K       & \si{10^3} \\
  centi & c       & \si{10^-2} \\
  milli & m       & \si{10^-3} \\
  micro & μ       & \si{10^-6} \\
  nano  & n       & \si{10^-9} \\ \bottomrule
\end{tabular}
\end{table}

\subsection{Significant Figures}
Significant Figures define a set of rules for maintaing percision while doing
math.  This system ensures that the answer to a problemset accurately reflects
the degree of percision of all sub-compounds.

\begin{enumerate}
  \item Zeros between non-zero digits are considered significant
  \item Zeros in front of a number are not significant
  \item Zeros to the end of a number and to the right of a decimal place are
    considered significant
  \item Zeros to the right of non-zeros \textit{may} be considered significant.
    \si{1000} has one digit of significance, while \si{1000.} has four.
    Similarly, \si{1.00e3} has three digits of significance.
\end{enumerate}

\subsection{Rules for Arithmetic}
When using digits of significance in arithmetic problems, it is important to
maintain a correct number of significant digits in the answer.  When adding and
subtracting numbers, the result has the minimum of the digits of significance
between the two products.  The same rules apply to both multiplication and
division.
