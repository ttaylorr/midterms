\chapter{Nation Building, Revolution}
\section{French and Indian War}
The French and Indian war was a war fought over territory in North America.  It
resulted in a shuffling of territories, with the real losers being the Native
Americans.  They were pushed out of their homeland, and that land was declared
to be for expansion purposes of the colonies.

\subsection{Pontiac's Rebellion, Proclimation Act}
However, following Pontiac's rebellion, the "Proclimation Act" was concieved,
preventing the colonists from settling territory west of the Appalachian
mountains.  Despite this, colonists settled anyway, setting up an attitude that
would promote the American revolution.

\section{British Policies}
Britain began enforcing a series of policies that made the colonists
not-so-happy.  They were also key players in the events following the American
Revolution.

\subsection{The Sugar Act}
The Sugar Act lowered existsing taxes on sugar imported to North America.
However, the act sought to crack down on the high rate of smuggling, thereby
strengthing the vice-admiralty courts system, which made the colonists upset.

\subsection{The Stamp Act}
The Stamp Act sought the place stamps (which cost money, hence "tax") on a
\textit{lot} of things.  This was purely designed to raise revenue.  At this
time the British stationed troops in Boston, forcing the locals to house and
feed these troops.

\subsubsection{The Stamp Act Congress}
These individuals, held up a congress with the goal of producing a document that
outlined their grievences to the British representatives.  They felt that they
were being taxed unjustly, as a result of a lack of representation.  Britain
believed in \textit{virtaul representation}, where as the colonists sought
\textit{direct representation}.

\begin{description}
  \item[Virtual Representation] a form of representative government where every
    elected official to the parlimentary office is seen as a representative of
    the whole nation, not a particular section of it.   This government is
    less effective at getting what a nation wants on a micro-scale, but moves
    much more quickly than a directly representative governemnt.
  \item[Direct Representation] a differing form of representative government,
    where each elected individual is responsible for representing a certain area
    of the nation, not its entirety.  This government is more effective at
    getting what the individuals want in a micro-sense, but acts much slower.
\end{description}

\subsubsection{The Committees of Correspondence}
These organizations spread inflamatory information regarding the British
government, acting as a sort-of foreshadow to the American Revolution.

\subsubsection{Crowd Actions}
Sons of Liberty groups harrassed, sometimes even attacking officers of the Stamp
Act.

\subsection{The Townshend Acts}
These imposed additiional taxes on the colonists, on paint, paper, lead, and
tea.

\subsubsection{The Boston Massacre}
