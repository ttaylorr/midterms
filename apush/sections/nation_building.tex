\chapter{Nation Building, Revolution}
\section{French and Indian War}
The French and Indian war was a war fought over territory in North America.  It
resulted in a shuffling of territories, with the real losers being the Native
Americans.  They were pushed out of their homeland, and that land was declared
to be for expansion purposes of the colonies.

\subsection{Pontiac's Rebellion, Proclimation Act}
However, following Pontiac's rebellion, the "Proclimation Act" was concieved,
preventing the colonists from settling territory west of the Appalachian
mountains.  Despite this, colonists settled anyway, setting up an attitude that
would promote the American revolution.

\section{British Policies}
Britain began enforcing a series of policies that made the colonists
not-so-happy.  They were also key players in the events following the American
Revolution.

\subsection{The Sugar Act}
The Sugar Act lowered existsing taxes on sugar imported to North America.
However, the act sought to crack down on the high rate of smuggling, thereby
strengthing the vice-admiralty courts system, which made the colonists upset.

\subsection{The Stamp Act}
The Stamp Act sought the place stamps (which cost money, hence "tax") on a
\textit{lot} of things.  This was purely designed to raise revenue.  At this
time the British stationed troops in Boston, forcing the locals to house and
feed these troops.

\subsubsection{The Stamp Act Congress}
These individuals, held up a congress with the goal of producing a document that
outlined their grievences to the British representatives.  They felt that they
were being taxed unjustly, as a result of a lack of representation.  Britain
believed in \textit{virtaul representation}, where as the colonists sought
\textit{direct representation}.

\begin{description}
  \item[Virtual Representation] a form of representative government where every
    elected official to the parlimentary office is seen as a representative of
    the whole nation, not a particular section of it.   This government is
    less effective at getting what a nation wants on a micro-scale, but moves
    much more quickly than a directly representative governemnt.
  \item[Direct Representation] a differing form of representative government,
    where each elected individual is responsible for representing a certain area
    of the nation, not its entirety.  This government is more effective at
    getting what the individuals want in a micro-sense, but acts much slower.
\end{description}

\subsubsection{The Committees of Correspondence}
These organizations spread inflamatory information regarding the British
government, acting as a sort-of foreshadow to the American Revolution.

\subsubsection{Crowd Actions}
Sons of Liberty groups harrassed, sometimes even attacking officers of the Stamp
Act.

\subsection{The Townshend Acts}
These imposed additiional taxes on the colonists, on paint, paper, lead, and
tea.

\subsection{The Tea Act}
Passed in 1773, the Tea Act eliminated British tariffs from tea sold in by the
British East India Company.  Though this lowered tea prices, Britain was accused
of doing special favors for large companies.  This resulted in the Boston Tea
Party dumping a lot of tea into the harbor.

\subsection{The Coercive/Intolerable Acts}
\begin{description}
  \item[The Massachusetts Government Act] brought the governance of
    Massachusetts under direct British control
  \item[The Administration of Justice Act] allowed authority to move trials from
    Massachusetts to Great Britain
  \item[The Boston Port Act] closed the port of Boston to trade
  \item[The Quartering Act] required Bostonian residents to house British troops
    upon command
\end{description}

\section{The War for Independence}
Fighting began in 1775 between colonists and British troops at Lexington and
Concord.  See: \textit{"the shot heard round the world"}.

\subsection{Factors in the Outcome of the War}
\begin{itemize}
  \item Britain had a large army, a fierce naval force, and a wealth of both
    funding and experience.  However, Britain was far from home, it had enemies,
    and its formal style of fighting was not well suited for the North American
    landscape.
  \item The Patriots had a great leader in George Washington, and support from
    several talented European generals.  Many Patriot soldiers believed in
    independence, but they did not have a lot of funding, nor a strong, central
    government.
\end{itemize}

\section{Foreign Policy of the Nation}
The United States now has the disadvantage of being surrounded by a lot of
unfriendly neighbors.  Spain, among others (as a result of the French and Indian
War) challenged America for land.  The question of alliance frequently became
discussed.  One school of thought was that the United States owed it to France
(and other Euro-powers) to help them since they were helped during the
revolution.  However, others felt that such relationships would entagle the
United States in world conflict.

\subsection{Washington and Neutrality}
Washington warned that neutrality was essential to the survival of the nation.
During this time, the XYZ affiar had occured, in which delegates were sent to
France to discuss a treaty.  They were told that discussions would not be had
until the United States paid them, and then offered them a loan.  No
converstaion was had with the nation.

\section{Enlightenment Philosophy}
\subsection{Ideas of John Locke}
John Locke published \textit{Two Treatises on Government}, in which he outline
two major ideas regarding governemnt that would later influence the
constitution.  He said that the govnerment was responsible for protecting the
natural rights of people, which were:

\begin{enumerate}
  \item Life
  \item Liberty
  \item Property
\end{enumerate}

He also argued that the consent of the government rests in the governed.  If the
government should fail to do the above, then it is the \textit{right} of the
people to overthrow that government via peaceful or violent rebellion.

\subsection{The Olive Branch Petition}
Congress sent the \textit{Olive Branch Petition} to parliment, asking for peace
between the two nations, but it was not signed.

\subsection{Common Sense}
Thoams Paine published \textit{Common Sense} in which he suggested that
Americans declare independence from Great Britain, because he could not "see a
single advantage" to staying with them.

\section{The Declaration of Independence}
Independence was declared!  It stated that "all men are created equal" and
"endowed by their Creator" with "certain unalienable rights".

\subsection{State Constitutions}
Several states had been encouraged to write up a state constitution, and several
did.  The ones that didn't updated their charters, but Pennsylvania remained
stubborn (more info needed).

\subsection{The Articles of Confederation}
The Articles of Confederation put down on paper what had been sort of assumed
over the past few years.  It was weak, but not ineffective.  The structure of
the government was a one-house legislature, where each state had two or seven
delegates, but each state only got one vote.
